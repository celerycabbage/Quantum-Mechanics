\documentclass{book}
\usepackage{amsmath}
\usepackage[UTF8]{ctex}
\usepackage{indentfirst}
\usepackage{enumerate}
\usepackage{graphicx}
\usepackage{wrapfig}
\usepackage{amssymb}
\usepackage{float}
\usepackage{subfigure}
\usepackage{cite}
\usepackage{caption}
\usepackage{setspace}
\usepackage{fancyhdr}
\usepackage{lastpage}
\usepackage{layout}
\usepackage{geometry}
\usepackage{tikz,mathpazo}
\newtheorem{definition}{\hspace{2em}定义}[section]
\newtheorem{theorem}{\hspace{2em}定理}[section]
\newtheorem{lemma}{\hspace{2em}引理}[section]
\newtheorem{proof}{证明}[section]
\newtheorem{example}{例}[section]
\newtheorem{corollary}{推论}[section]
\begin{document}
\title{Quantum Mechanics}
\author{大白菜}
\date{2018-12-13}
\maketitle
\tableofcontents
\part{非相对论量子力学}
\chapter{线性代数}
\section{简介}
量子力学最主要的数学工具是线性代数. 要学好量子力学, 必须非常长熟悉线性代数中的基本概念和运算方法. 虽然我们假定读者已经学习过线性代数, 但考虑到不同读者采用的符号标记不同, 练习不够, 甚至接受了某些错误的概念, 有时会对量子力学学习造成一些不必要的麻烦. 因此本书之初, 我们结合物理学, 复习线性代数中的基本概念和运算方法, 统一符号, 强调某些容易混淆的概念, 我们愿意提醒读者, 理解这些概念和运算方法, 并不等于能熟练使用它们.
\section{向量空间和子空间}
\subsection*{向量空间和向量空间的基}
\indent如果数集$F$中任意两个数作某一运算后的结果仍然在$\mathbf{F}$中, 我们就称数集$\mathbf{F}$对这个运算时封闭的, 对加减乘除四则运算封闭的数集$\mathbf{F}$称为数域.

\indent最常见的数域是有理数域$\mathbf{Q}$、实数域$\mathbf{R}$、复数域$\mathbf{C}$.此外还有很多其他数域, 比如$\mathbf{Q}(\sqrt{2})$:
\begin{equation*}
  \mathbf{Q}(\sqrt{2})=\{a+b\sqrt{2}|a,b\in\mathbf{Q}\}
\end{equation*}
\begin{definition}[向量空间]
  设$E$是一个非空集合, $\mathbf{F}$是一个数域, 在集合$E$的元素之间定义了加法.在数域$\mathbf{F}$与集合$E$的元素之间还定义了数量乘法, 如果加法与数量乘法满足下述规则:
  \begin{enumerate}[(1)]
    \item 加法交换律\\$a+b=b+a$
    \item 加法结合律\\$(a+b)+c=a+(b+c)$
    \item 在$E$中有单位元$\mathbf{0}$\\$a+0=a$
    \item 对于$E$中每一个元素$a$都存在逆元素$b$\\$a+b=\mathbf{0}$
    \item $1a=a$
    \item $k(la)=(kl)a$
    \item $(k+l)a=ka+la$
    \item $k(a+b)=ka+kb$
  \end{enumerate}
  则称$E$为数域$\mathbf{F}$上的向量空间, 其中$k,l\in \mathbf{F}$,而$a,b\in E$,其中的元素也常称为向量,数域$F$中的元素也常称为标量.
\end{definition}
\indent例如矩阵元是$\mathbf{F}$中标量构成的矩阵, 对于矩阵加法和矩阵的数乘, 构成数域$\mathbf{F}$上的向量空间.全体实函数, 按照函数加法与乘法, 也构成实数域上的向量空间
\begin{definition}[向量空间的基]
  若向量空间$E$中, 有$n$个向量$a_1,a_2,a_3,\cdots,a_n$线性无关,而$E$中任意$n+1$个向量线性相关, 则把这$a_1,a_2,\cdots,a_n$称为向量空间$E$的一组基底, 由于向量空间中的所有基底都含有相同数量的向量, 称$n$为向量空间$E$的维数.记作:$dimE=n$
\end{definition}
\subsection*{向量空间的子空间}
\begin{definition}[子空间]
  如果数域$\mathbf{F}$上的向量空间$E$的非空子集$E_1,E_2$对于$E$的两种运算也构成向量空间,则称$E_1,E_2$为$E$的一个子空间
\end{definition}
\indent例如$\mathbb{W}=\{\begin{pmatrix}
                       a & b \\
                       0 & 0 \\
                     \end{pmatrix}
|a,b\in \mathbf{F}\}$是$F^{2\times 2}$的一个子空间.

\indent显然向量空间中, 由单个的零向量所构成的子集合是一个向量空间, 它叫做\textbf{零子空间}.

\indent向量空间$E$自己本身也是$E$的一个子空间.

\indent在向量空间总, 零子空间和向量空间本身这两个子空间有时候叫做\textbf{平凡子空间}, 而其他线性子空间叫做\textbf{非平凡子空间}.
\begin{definition}[子空间的和]
  设$E_1,E_2$是向量空间$E$的两个子空间.把所有能表示成$a_1+a_2(a_1\in E_1,a_2\in E_2)$的向量组成的子集合叫做$E_1$和$E_2$的和, 记作:$E_1+E_2$
\end{definition}
\indent很容易看出, 子空间的和$E_1+E_2$与子空间的交$E_1\bigcap E_2$也是都是子空间
\begin{definition}
  设$E_1,E_2$是向量空间$E$的两个子空间.
  \begin{equation*}
    dim(E_1)+dim(E_2)=dim(E_1+E_2)+dim(E_1\cap E_2)
  \end{equation*}
\end{definition}
\indent子空间和的维数一般比子空间维数之和要小.
\begin{definition}[子空间的直和]
  设$E_1,E_2$是向量空间$E$的两个子空间. 如果$E=E_1+E_2$并且$E_1\cap E_2=\{0\}$.则把$E$叫做$E_1,E_2$的直和, 记作$E=E_1\oplus E_2$.其中我们把$E_1,E_2$叫做$E$的互补子空间.
\end{definition}

\indent直和有三条等价条件
\begin{enumerate}[(1)]
  \item $E_1\cap E_2=\{0\}$
  \item $dim(E)=dim(E_1)+dim(E_2)$
  \item $E$中任意向量都可以唯一地分解成分别属于$E_1,E_2$中的向量之和
\end{enumerate}
\indent下面是几个向量空间的例子:
\begin{example}[函数空间]\label{function space}
  若$X$是一个非空集合, $E$是一个向量空间.记$F$是$X\longmapsto E$的所有函数构成的空间.如果加法和标量乘法满足下面的规则:
  \begin{enumerate}[(1)]
    \item $(f+g)(x)=f(x)+g(x)$
    \item $(\lambda f)(x)=\lambda f(x)$
  \end{enumerate}
  则$F$构成向量空间, 其中$0$矢量是对于$X$中每一个元素$x$都满足$f(x)=0$,映射到$\mathbb{E}$中的$0$向量所对应的映射.
\end{example}
\begin{example}
  $\Omega$是$R^N$的一个开子集, 所有$\Omega\longmapsto\mathbb{C}$的函数, 满足下列条件构成子空间:
  \begin{enumerate}[(1)]
    \item $\mathcal{C}(\Omega)=$定义在$\Omega$上的所有连续复函数的空间
    \item $\mathcal{C}^k(\Omega)=$定义在$\Omega$上的所有k阶连续可偏导复函数的空间
    \item $\mathcal{C}^\infty(\Omega)=$定义在$\Omega$上的所有无穷阶可微复函数的空间
    \item $\mathcal{P}(\Omega)=$定义在$\Omega$上的所有$N$元多项式的空间
  \end{enumerate}
\end{example}
\begin{example}[序列空间]
  如果在\ref{function space}中的非空集合$X$是正整数集, 那么对应的函数空间叫做\textbf{序列空间}.加法和标量乘法如下定义:
  \begin{enumerate}[(1)]
    \item $(x_1,x_2,\dots)+(y_1,y_2,\dots)=(x_1+y_1,x_2+y_2,\dots)$
    \item $\lambda(x_1,x_2,\dots)=(\lambda x_1,\lambda x_2,\dots)$
  \end{enumerate}
  所有复序列构成的空间是一个向量空间.有界复序列的空间是这个空间的真子空间.收敛复序列的空间是有界序列空间的真子空间
\end{example}
\indent我们用$(x_n)$或者$(x_1,x_2,\dots)$来表示序列的第$n$项$x_n$,用$\{x_n:n\in\mathbb{N}\}$去表示序列元素的集合.值得注意的是, 尽管$(x_n)$是无穷序列, 但$\{x_n:n\in\mathbb{N}\}$可以是有限集合.
\indent在大多数情况下, 证明一个空间是向量空间是容易的或者平凡的.下面给两个例子.
\begin{example}[$l^p$空间]
  无穷复序列$(z_n)$满足$\sum_{n=1}^{\infty}|z_n|^p<\infty$的空间记作\textbf{$\mathbf{l^p}$空间}$(p\geq1)$
\end{example}
\begin{proof}
  \indent$l^p$空间是复序列空间的子空间, 如果$(x_n),(y_n)\in l^p$, 很容易得到$(x_n+y_n)\in l^p$和$(\lambda x_n)\in l^p$

  \indent首先显然
  \begin{equation*}
    \sum_{n=1}^\infty|\lambda x_n|^p=|\lambda|^p\sum_{n=1}^{\infty}|x_n|^p<\infty
  \end{equation*}
而条件$\sum_{n=1}^{\infty}|x_n+y_n|<\infty$则可以用Minkowski不等式得到:
\begin{equation*}
  (\sum_{n=1}^{\infty}|x_n+y_n|^p)^{\frac{1}{p}}\leq(\sum_{n=1}^{\infty}|x_n|^p)^{\frac{1}{p}}+(\sum_{n=1}^{\infty}|y_n|^p)^{\frac{1}{p}}
\end{equation*}
\end{proof}
Minkowski不等式建立于H\"{o}lder不等式.读者可以自证,以下是参考证明:
\begin{theorem}[H\"{o}lder不等式]
  设$p>1,q>1$并且有$\frac{1}{p}+\frac{1}{q}=1$, 如果$(x_n)\in l^p$以及$(y_n)\in l^p$, 那么有:
  \begin{equation*}
    \sum_{n=1}^{\infty}|x_ny_n|\leq(\sum_{n=1}^{\infty}|x_n|^p)^{\frac{1}{p}}(\sum_{n=1}^{\infty}|y_n|^q)^{\frac{1}{q}}
  \end{equation*}
\end{theorem}
\begin{proof}
  不失一般性, 我们假设$\sum_{n=1}^{\infty}|x_n|\neq0,\sum_{n=1}^{\infty}|y_n|\neq0$. 首先我们观察:
  \begin{equation*}
    x^{\frac{1}{p}}\leq\frac{1}{p}x+\frac{1}{q}
  \end{equation*}
  对于$0<x<1$成立.设$a,b$是非负数, 且有$a^p\leq b^q$. 这样就有$0\leq\frac{a^p}{b^q}\leq1$. 因此我们有:
  \begin{equation*}
    ab^{-\frac{q}{p}}\leq\frac{1}{p}\frac{a^p}{b^q}+\frac{1}{q}
  \end{equation*}
  因为$-\frac{q}{p}=1-q$.
  \begin{equation*}
    ab^{1-q}\leq\frac{1}{p}\frac{a^p}{b^q}+\frac{1}{q}
  \end{equation*}
  两边同时乘以$b^q$, 得到
  \begin{equation}\label{holder inequality}
    ab\leq\frac{a^p}{p}+\frac{b^q}{q}
  \end{equation}
  证明\ref{holder inequality}用了假设条件$a^p\leq b^q$.同理可以证明\ref{holder inequality}同样对于$a^p\geq b^q$也成立. 对于\ref{holder inequality}我们假设:
  \begin{equation*}
    a=\frac{x_j}{(\sum_{k=1}^{n}|x_k|^p)^\frac{1}{p}}\quad\quad  b=\frac{y_j}{(\sum_{k=1}^{n}|y_k|^q)^\frac{1}{q}}
  \end{equation*}
  其中$n\in\mathbb{N}$,我们得到
  \begin{equation*}
    \frac{x_j}{(\sum_{k=1}^{n}|x_k|^p)^{\frac{1}{p}}}\frac{y_j}{(\sum_{k=1}^{n}|y_k|^q)^\frac{1}{q}}\leq\frac{1}{p}\frac{|x_j|^p}{\sum_{k=1}^{n}|x_k|^p}+\frac{1}{q}\frac{|y_j|^q}{\sum_{k=1}^{n}|y_k|^q}
  \end{equation*}
  对于$1\leq j\leq n$.叠加上式, 得到
  \begin{equation*}
    \frac{\sum_{k=1}^{n}|x_j||y_j|}{(\sum_{k=1}^{n}|x_k|^p)^{\frac{1}{p}}(\sum_{k=1}^{n}|y_k|^q)\frac{1}{q}}\leq\frac{1}{p}+\frac{1}{q}=1
  \end{equation*}
  让$n\rightarrow\infty$得到H\"{o}lder不等式
\end{proof}
\begin{theorem}[Minkowski不等式]
  设$p\geq1$. 如果$(x_n),(y_n)\in l^p$, 那么
  \begin{equation*}
    (\sum_{n=1}^{\infty}|x_n+y_n|^p)^{\frac{1}{p}}\leq(\sum_{n=1}^{\infty}|x_n|^p)^\frac{1}{p}+(\sum_{n=1}^{\infty}|y_n|^p)^\frac{1}{p}
  \end{equation*}
\end{theorem}
\begin{proof}
  对于$p=1$很显然退化成三角不等式. 如果$p>1$, 那么存在$q$满足$\frac{1}{p}+\frac{1}{q}=1$. 根据H\"{o}lder不等式, 有:
  \begin{equation*}
  \begin{split}
     \sum_{n=1}^{\infty}|x_n+y_n|^p&\leq\sum_{n=1}^{\infty}|x_n+y_n||x_n+y_n|^{p-1}\\
       &\leq\sum_{n=1}^{\infty}|x_n||x_n+y_n|^{p-1}+\sum_{n=1}^{\infty}|y_n||x_n+y_n|^{p-1}\\
       &\leq(\sum_{n=1}^{\infty}|x_n|^p)^\frac{1}{p}(\sum_{n=1}^{\infty}|x_n+y_n|^{q(p-1)})^\frac{1}{q}\\
       &+(\sum_{n=1}^{\infty}|y_n|^p)^\frac{1}{p}(\sum_{n=1}^{\infty}|x_n+y_n|^{q(p-1)})^\frac{1}{q}
  \end{split}
  \end{equation*}
  因为$q(p-1)=p$,所以有:
  \begin{equation*}
    \sum_{n=1}^{\infty}|x_n+y_n|^p\leq ((\sum_{n=1}^{\infty}|x_n|^p)^\frac{1}{p}+(\sum_{n=1}^{\infty}|y_n|^p)^\frac{1}{p})(\sum_{n=1}^{\infty}|x_n+y_n|^{p})^\frac{1}{q}
  \end{equation*}
  两边同时除以$(\sum_{n=1}^{\infty}|x_n+y_n|^p)^{\frac{1}{q}}$后, Minkowski不等式得证.
\end{proof}
\begin{definition}[向量空间的笛卡尔乘积]
  设$E_1,\dots,E_n$是在相同数域$\mathbf{F}$上的向量空间, 定义:
  \begin{equation*}
    E={(x_1,\dots,x_n):x_1\in E_1,x_2\in E_2,\dots,x_n\in E_n}
  \end{equation*}
  并且有:
  \begin{equation*}
    (x_1,\dots,x_n)+(y_1,\dots,y_n)=(x_1+y_1,\dots,x_n+y_n)
  \end{equation*}
  \begin{equation*}
    \lambda(x_1,\dots,x_n)=(\lambda x_1,\dots,\lambda x_n)
  \end{equation*}
  $E$是一个向量空间, 叫做$E_1,\dots,E_n$的笛卡尔乘积.记作$E=E_1\times E_2\times\dots\times E_n$
\end{definition}
设$x_1,\dots,x_n$是向量空间$E$的元素.如果存在标量$\alpha_1,\dots,\alpha_k$满足:
\begin{equation*}
  x=\alpha_1x_1+\dots+\alpha_kx_k
\end{equation*}
则向量$x\in E$叫做向量$x_1,\dots,x_n$的线性组合.例如$R^n$中的元素是
\begin{equation*}
  e_1=(1,0,\dots,0),\quad e_2=(0,1,\dots,0),\dots,\quad e_n=(0,0,\dots,0,1)
\end{equation*}
的线性组合.

\indent类似的, k阶多项式是$1,x,x^2,\dots,x^k$的线性组合

\begin{definition}[线性无关]
  如果$\alpha_1x_1+\dots+\alpha_kx_k=0$当且仅当$\alpha_1=\alpha_2=\dots=\alpha_k=0$, 那么矢量${x_1,x_2,\cdots,x_k}$线性无关
\end{definition}
\section{赋范空间}
\subsection*{范数}
\begin{definition}[范数]
  向量空间$E\mapsto\mathbb{R}$的函数$x\mapsto\|x\|$被称为范数, 如果满足下列条件:
  \begin{enumerate}[(a)]
    \item $\|x\|=0$,当且仅当$x=0$
    \item 对于任意$x\in E,\lambda\in F$,有$\|\lambda x\|=|\lambda|\|x\|$
    \item 对于任意$x,y\in E$,有$\|x+y\|\leq\|x\|+\|y\|$
  \end{enumerate}
\end{definition}
条件$(c)$叫做三角不等式
\begin{example}
  函数
  \begin{equation*}
    \|z\|=\sqrt{|z_1|^2+|z_2|^2+\dots+|z_n|^2},\quad z=(z_1,\dots,z_n)\in\mathbb{C}^n
  \end{equation*}
  定义了$\mathbb{C}^n$上的一个范数. 这个范数叫做$Euclidean$范数. 下列同样也是$\mathbb{C}^n$的范数:
  \begin{equation*}
    \begin{split}
       \|z\| & =|z_1|+|z_2|+\dots+|z_n| \\
       \|z\| & =max{|z_1|,\dots,|z_n|}
    \end{split}
  \end{equation*}
\end{example}
\begin{example}\label{norm1}
  设$\Omega$是$\mathbb{R}^n$上的有界闭子集. 函数$\|f\|=max_{x\in\Omega}|f(x)|$定义了$\mathcal{C}(\Omega)$上的范数
\end{example}
\begin{example}\label{norm2}
  设$z=(z_n)\in l^p$. $\|z\|=(\sum_{n=1}^{\infty}|z_n|^p)^\frac{1}{p}$定义了$l^p$上的一个范数$(p>1)$, Minkowski不等式保证了这一点.
\end{example}
\subsection*{赋范空间}
\begin{definition}
  具有范数的向量空间叫做赋范空间
\end{definition}
在同一个向量空间上可能定义不同的范数. 因此定义赋范空间需要具体的向量空间和范数定义. 我们也常说赋范空间指的是一对$(E,\|\cdot\|)$, 其中$E$是指向量空间, $\|\cdot\|$是指定义在$E$上的范数. 一些向量空间也具有标准的范数. 但我们谈论到$R^n$的时候, 我们立即意识到$Euclidean$范数:
\begin{equation*}
  \|x\|=\sqrt{x_1^2+\dots+x_n^2}
\end{equation*}
同样类似的, \ref{norm1}、\ref{norm2}里定义的范数也是标准的. 如果我们想要考虑这些空间上的不同范数, 我们就必须说清楚“考虑$\dots$上具有$\dots$范数的空间”.

\indent指的注意, 赋范空间的子空间也要有相同的范数定义.
\indent绝对值是$R$和$C$上的一个范数. 因为两个数差的绝对值表示是这些数的距离, 并且收敛是指"与极限点的接近", 所以绝对值通常用于定义收敛. 范数也扮演同样的角色. 当$\|x\|$被解释为$x$的大小时, $\|x-y\|$提供了$x$和$y$之间的距离尺度.
\subsection*{赋范空间中的收敛}
\begin{definition}
  设$(E,\|\cdot\|)$是一个赋范空间, 如果对于任意$\varepsilon>0$, 存在一个数$M$, 使得对于每一个$n\geq M$都有$\|x_n-x\|<\varepsilon$, 我们说$E$中元素序列$(x_n)$收敛于$x\in E$. 在一些情况下, 可以写成$\lim_{n\to \infty}=x$或者简单写作$x_n\to x$
\end{definition}
赋范空间上的收敛有和$\mathbb{R}$上收敛的相同性质:
\begin{enumerate}
  \item 如果$x_n\to x$和$\lambda_n\to\lambda$($\lambda_n,\lambda$是数域里的标量), 那么有$\lambda_n x_n\to\lambda x$
  \item 如果$x_n\to x$和$y_n\to y$, 那么有$x_n+y_n\to x+y$
\end{enumerate}
证明方法与$\mathbb{R}$中收敛的方法一样.

\indent 向量空间$E$中的范数诱导向量空间$E$的收敛. 换句话说, 如果我们有一个赋范空间$E$,我们自然地就在$E$中定义了收敛. 很自然的我们会想到, 反过来, 如果给定了$E$上的收敛, 是否可以找到$E$上的范数, 这个范数可以定义这个收敛. 答案是不一定.
\begin{example}
  考虑定义在有界闭集$\Omega\subset \mathbb{R}^n$上的所有连续函数空间$\mathcal{C}(\Omega)$. 若对于任意$\varepsilon>0$, 存在一个常数$n_0$, 使得对于所有$x\in\Omega$和$n\geq n_0$都有$|f(x)-f_n(x)|<\varepsilon$成立, 我们说序列$f_1,f_2,\dots,f_n\in\mathcal{C}(\Omega)$一致收敛于$f$. 显然\ref{norm1}中的范数定义了序列$f_n$一致收敛于$f$, 当且仅当$\|f_n-f\|=\max_{x\in\Omega}|f_n(x)-f(x)|\to 0(n\to 0)$. 这个范数叫做一致收敛范数
\end{example}
\subsection*{范数的等价}
\begin{definition}
  定义在相同向量空间的两个范数, 如果他们定义相同的收敛, 那么被称为等价范数. 更详细的说, 定义在$E$上的范数$\|\cdot\|_1$和$\|\cdot\|_2$, 如果对于在$E$上的任意序列$(x_n)$和$x\in E$都有,
  \begin{equation*}
    \|x_n-x\|_1\to 0 \quad \text{当且仅当}\quad \|x_n-x\|_2\to 0
  \end{equation*}
  那么被称为等价的.
\end{definition}
\begin{example}
  下列定义在$R^2$上的范数是等价的
  \begin{equation*}
  \begin{split}
       &\|(x,y)\|_1=\sqrt{x^2+y^2}\quad \|(x,y)\|_2=|x|+|y| \\
       &\|(x,y)\|_3=\max{|x|,|y|}
  \end{split}
  \end{equation*}
\end{example}
接下来将会证明定义在任意有限维向量空间的范数都是等价的. 证明需要一些准备.
\begin{theorem}\label{norm equivalence}
  设$\|\cdot\|_1$和$\|\cdot\|_2$是向量空间的两个范数. $\|\cdot\|_1$和$\|\cdot\|_2$被称为等价的, 当且仅当存在正数$\alpha$和$\beta$, 使得
  \begin{equation*}
    \alpha\|x\|_1\leq\|x\|_2\leq\beta\|x\|_1\quad x\in E
  \end{equation*}
\end{theorem}
\begin{proof}
  假设两个范数$\|\cdot\|_1$和$\|\cdot\|_2$等价, 也就是说$\|x_n-x\|_1\to 0$当且仅当$\|x_n-x\|_2\to 0$. 假设不存在$\alpha>0$使得对于任意$x\in E$都有$\alpha\|x\|_1\leq\|x\|_2$. 那么对于任意$n\in \mathbb{N}$都存在$x_n\in E$使得
  \begin{equation*}
    \frac{1}{n}\|x_n\|_1>\|x\|_2
  \end{equation*}
  定义
  \begin{equation*}
    y_n=\frac{1}{\sqrt{n}}\frac{x_n}{\|x_n\|_2}
  \end{equation*}
  显然$\|y_n\|_2=\frac{1}{\sqrt{n}}\to 0$. 另一方面, $\|y_n\|_1\geq n\|y_n\|_2\geq \sqrt{n}$. 这个矛盾表明了必须存在$\alpha>0$满足定理左半边. 同样的可以证明$\beta$的存在.
\end{proof}
\begin{lemma}\label{norm Lemma}
  如果$x_1,\dots,x_n$是赋范空间中线性无关的元素, 则存在常数$c>0$使得
  \begin{equation}\label{norm lemma}
    \|\alpha_1x_1+\dots+\alpha_nx_n\|\geq c(|\alpha_1|+\dots+|\alpha_n|)
  \end{equation}
  对于所有的$\alpha_1,\dots,\alpha_n\in\mathbb{R}$都成立
\end{lemma}
\begin{proof}
  当$|\alpha_1|+\dots+|\alpha_n|=0$时,\eqref{norm lemma}对与任意$c$都成立. 所以\eqref{norm lemma}等价于
  \begin{equation*}
    \|\beta_1x_1+\dots+\beta_nx_n\|\geq c,\quad|\beta_1|+\dots+|\beta_n|=1
  \end{equation*}
  连续函数$f:\mathbb{R}^n\mapsto\mathbb{R}$定义为:
  \begin{equation*}
    f(\beta_1,\dots,\beta_n)=\|\beta_1x_1+\dots+\beta_nx_n\|
  \end{equation*}
  因为集合$B={(\beta_1,\dots,\beta_n)\in\mathbb{R}^n:|\beta_1|+\dots+|\beta_n|=1}$是有界闭集. $f$在$B$上取得最小值. 值得注意最小值非零, 因为$\beta_1x_1+\dots+\beta_nx_n=0,|\beta_1|+\dots+|\beta_n|=1$与$x_1,\dots,x_n$线性无关矛盾.因此
  \begin{equation*}
    c=\min_{(\beta_1,\dots,\beta_n)\in B}f(\beta_1,\dots,\beta_n)=\min_{(\beta_1,\dots,\beta_n)\in B}\|\beta_1x_1+\dots+\beta_nx_n\|>0
  \end{equation*}
  证明完毕
\end{proof}
\begin{theorem}
  如果$E$是有限维向量空间, 则该空间任意两个范数都等价.
\end{theorem}
\begin{proof}
  $E$是有限维向量空间,$\{e_1,\dots,e_n\}$是$E$的基.定义$\|\cdot\|_0$范数如下:
  \begin{equation*}
    \|\alpha_1e_1+\dots+\alpha_ne_n\|_0=|\alpha_1|+\dots+|\alpha_n|
  \end{equation*}
  利用\ref{norm equivalence}
  \begin{equation*}
  \begin{split}
     \|\alpha_1e_1+\dots+\alpha_ne_n\|&\leq|\alpha_1|\|e_1\|+\dots+|\alpha_n|\|e_n\| \\
       &\leq\max{\|e_1\|,\dots,\|e_n\|}(|\alpha_1|+\cdots+|\alpha_n|)
  \end{split}
  \end{equation*}
  所以对于任意$x\in E$都有
  \begin{equation*}
    \|x\|\leq\beta\|x\|_0
  \end{equation*}
  其中$\beta=\max{\|e_1\|,\dots,\|e_n\|}$. 由于引理\ref{norm Lemma}立即可以得到, 对于任意$x\in E$
  \begin{equation*}
    \alpha\|x\|_0\leq\|x\|
  \end{equation*}
  证明完毕
\end{proof}
如果定义距离函数$d(x,y)=\|x-y\|$, 则赋范空间$(E,\|\cdot\|)$成为度规空间.由范数定义的收敛和由这个度规定义的收敛是一样的. $E$中的度规定义了$E$中的拓扑. 拓扑概念可以通过开集去定义,没有必要先定义度规, 然后用度规去定义拓扑.下面我介绍一些赋范空间的拓扑性质.
\subsubsection*{开球、闭球和球面}
\begin{definition}
  $x$是度量空间$E$里的元素, $r$是正数, 则有如下概念:
  \begin{enumerate}
    \item $B(x,y)={y\in E:\|y-x\|<r}$(开球)
    \item $B(x,y)={y\in E:\|y-x\|\leq r}$(闭球)
    \item $B(x,y)={y\in E:\|y-x\|=r}$(球面)
  \end{enumerate}
  $x$叫做球心, $r$叫做半径
\end{definition}
\begin{example}
  下面是几个$R^2$中球的例子.
  \begin{equation*}
    \begin{split}
       \|(x,y)\|_1 & =\sqrt{x^2+y^2},\quad\|(x,y)\|_2=|x|+|y| \\
       \|(x,y)\|_3 & =\max{|x|,|y|}
    \end{split}
  \end{equation*}
\begin{figure}[H]
  \centering
  \begin{minipage}{3cm}
    \includegraphics[width=3cm]{fig1.pdf}
  \end{minipage}
  \begin{minipage}{3cm}
    \includegraphics[width=3cm]{fig2.pdf}
  \end{minipage}
  \begin{minipage}{3cm}
    \includegraphics[width=3cm]{fig3.pdf}
  \end{minipage}
  \caption{$\mathbb{R}^2$中球的例子, $\|(x,y)\|_1\leq1, \|(x,y)\|_2\leq1, \|(x,y)\|_3\leq1$}
\end{figure}
\end{example}
\subsubsection*{开集和闭集}
\begin{definition}
  $S$是赋范空间$E$的子集, 若对于任意$x\in S$都存在$\varepsilon>0$, 使得$B(x,\varepsilon)\subseteq S$,则称$S$为开集. 如果子集$S$的补集$(E\setminus S)$是开集, 则称$S$为闭集.
\end{definition}
很容易意识到, 尽管球的定义不同, 但范数的等价定义了相同的开集. 开集同样的定义了闭集、稠集和紧集以及其他的拓扑概念.
\begin{example}
  开球是开集. 闭球和球面是闭集.
\end{example}
\begin{theorem}
\begin{enumerate}[(a)]
  \item 开集的交集是开集
  \item 开集的有限交集是开集
  \item 闭集的有限并集是闭集
  \item 闭集的交仍然是闭集
  \item 空集和全集既是开也是闭的
\end{enumerate}
\end{theorem}
\begin{theorem}
  赋范空间$E$的子集$S$是闭的, 当且仅当$S$的元素序列在$E$中收敛,并且极限在$S$中.
  \begin{equation*}
    x_1,x_2,\dots,x_n\in S,\quad x_n\to x,\quad x\in S
  \end{equation*}
\end{theorem}
\begin{proof}
  假设$S$是$E$的闭子集, $x_1,x_2,\dots\in S$, $x_n\to x$且$x\notin S$. 因为$S$是闭集, $E\setminus S$是开集. 所以存在$\varepsilon>0$使得$B(x,\varepsilon)\subseteq E\setminus S$. 换句话说,因为$\|x-x_n\|\to 0$, 对于充分大的$n\in \mathbb{N}$, 我们有$\|x-x_n\|<\varepsilon$. 这个矛盾表明了$x\in S$.

  \indent现在假设$x_1,x_2,\dots\in S$, 并且有$x_n\to x$,$x\in S$. 如果$S$不是闭集, 则$E\setminus S$非开. 所以存在$x\in E\setminus S$使得任意球$B(x,\varepsilon)$都包含$S$中的元素. 这样我们能找到$x_1,x_2,\dots\in S$使得$x_n\in B(x,\frac{1}{n})$. 但另一方面$x_n\to x$以及假设条件$x\in S$. 这与$x\in E\setminus S$假设相互矛盾. 因此$S$必须是闭集.
\end{proof}
\subsubsection*{闭包}
\begin{definition}
  $S$是赋范空间$E$的子集, $S$的闭包定义为所有闭集的交, 记作$cl S$.表示最小的闭集.
\end{definition}
\begin{theorem}
  $S$是赋范空间$E$的子集, 闭包$cl S$是指$S$中所有收敛列极限的集合.
  \begin{equation*}
    cl S=\{x\in E:\exists x_1,x_2,\dots\in S s.t. x_n\to x\}
  \end{equation*}
\end{theorem}
读者自证
\subsubsection*{稠密}
\begin{definition}
  赋范空间$E$有子集$S$, 如果$cl S=E$, 则$S$被称为稠密的,
\end{definition}
\begin{example}
  $[a,b]$上所有多项式的集合在$\mathcal{C}([a,b])$上稠密. 有限项非零项的所有复数列的集合在$l^p(p>1)$上稠密.
\end{example}
\begin{theorem}
  $S$是赋范空间$E$的子集, 以下条件相互等价:
  \begin{enumerate}[(a)]
    \item $S$在$E$中稠密
    \item 对于每一个$x\in E$, 都存在$x_1,x_2,\dots\in S$,使得$x_n\to x$
    \item 对于$E$的每一个非空子集至少包含$S$中的一个元素
  \end{enumerate}
\end{theorem}
\subsubsection*{紧致}
\begin{definition}
  赋范空间$E$有子集$S$, 如果$S$中的每一个序列$(x_n)$都包含极限在$S$中的收敛子序列, 则称$S$为紧致集合
\end{definition}
\begin{example}
  $R^n,C^n$中的有界闭集紧致
\end{example}
\begin{theorem}
  当且仅当闭单位球是紧致的时候, 赋范空间$E$是有限维的
\end{theorem}
\subsubsection*{有界}
\begin{definition}
  赋范空间$E$中有子集$S$, 如果存在$r>0$, $S\subseteq B(0,r)$, 则称$S$是有界子集
\end{definition}
\begin{theorem}
  紧致集合是有界闭的.
\end{theorem}
\begin{theorem}
  当且仅当$E$中的单位球是紧致的, 赋范空间$E$是有限维的.
\end{theorem}
读者自证
\section{Banach空间}
\subsection*{Cauchy序列}
\begin{definition}
  如果对于任意$\varepsilon>0$存在数$M$使得$\|x_m-x_n\|<\varepsilon$对于$m,n>M$都成立, 则称赋范空间中的向量序列$(x_n)$为Cauchy序列.
\end{definition}
Augustin Louis Cauchy(1789-1857)
\begin{theorem}
  下列条件相互等价
  \begin{enumerate}[(a)]
    \item $(x_n)$是Cauchy序列.
    \item 对于任意一对递增正整数序列$(p_n),(q_n)$, 都有$\|x_{p_n}-x_{q_n}\|\to0,n\to\infty$成立
    \item 对于任意递增正整数序列$(p_n)$, 都有$\|x_{p_{n+1}}-x_{p_n}\|\to0,n\to\infty$成立
  \end{enumerate}
\end{theorem}
显然每一个收敛序列都是Cauchy序列,事实上若$\|x_n-x\|\to0$, 则
\begin{equation*}
  \|x_{p_n}-x_{q_n}\|\leq\|x_{p_n}-x\|+\|x_{q_n}-x\|\to0
\end{equation*}
对于每一个递增序列$(p_n),(q_n)$都成立.收敛序列是Cauchy序列, 但Cauchy序列不一定是收敛序列.
\begin{example}
  设$\mathcal{P}([0,1])$是$[0,1]$上的多项式空间, 并且有一致收敛的范数$\|P\|=\max_{[0,1]}|P(x)|$. 定义
  \begin{equation*}
    P_n(x)=1+x+\frac{x^2}{2!}+\cdots+\frac{x^n}{n!}
  \end{equation*}
  其中$n=1,2,\dots$, $(P_n)$是一个Cauchy序列, 但显然并不收敛于$\mathcal{P}([0,1])$, 因为极限不是一个多项式.
\end{example}
\begin{lemma}
  若$(x_n)$是赋范空间的一个Cauchy序列, 则序列$(\|x_n\|)$收敛.
\end{lemma}
\begin{proof}
  因为$|\|x\|-\|y\||\leq\|x-y\|$, 所以有$|\|x_m\|-\|x_n\||\leq\|x_m-x_n\|\to0,(m,n\to \infty)$. 范数序列是实柯西序列, 因此它收敛.
\end{proof}
\subsection*{Banach空间}
\begin{definition}
  如果每一个Cauchy序列都收敛于$E$,则称$E$是完备的.完备的赋范空间称为Banach空间
\end{definition}
\begin{example}\label{l^2 complete}
  $l^2$是完备空间. 设$(a_n)$是$l^2$的柯西序列, 如果
  \begin{equation*}
    a_n=(\alpha_{n,1},\alpha_{n,2},\dots)
  \end{equation*}
  对于给定的任意$\varepsilon>0$都存在数$n_0$使得
  \begin{equation*}\label{banach l^2}
    \sum_{k=1}^{\infty}|\alpha_{m,k}-\alpha_{n,k}|^2<\varepsilon^2
  \end{equation*}
  对于所有的$m,n\geq n_0$都成立. 值得注意, 这实际上是对于每一个固定的$k\in \mathbb{N}$和每一个$\varepsilon$都存在一个数$n_0$使得
  \begin{equation*}
    |\alpha_{m,k}-\alpha_{n,k}|<\varepsilon
  \end{equation*}
  对于任意$mn,\geq n_0$都成立. 这也意味着对于任意$k$, 序列$(\alpha_{n,k})$是$\mathbb{C}$里的柯西序列, 所以是收敛序列.记
  \begin{equation*}
    \alpha_k=\lim_{n\to\infty}\alpha_{n,k},\quad k=1,2,\dots\quad a=(\alpha_{n})
  \end{equation*}
  进一步证明$a$是$l^2$的元素, 序列$(a_n)$收敛于$a$. 根据\eqref{banach l^2}, 令$m\to \infty$得到
  \begin{equation*}
    \sum_{k=1}^{\infty}|\alpha_k-\alpha_{n,k}|^2\leq\varepsilon^2
  \end{equation*}
  对于任意$n\geq n_0$都成立. 因为
  \begin{equation*}
    \sum_{k=1}^{\infty}|\alpha_{n_0,k}|^2<\infty
  \end{equation*}
  再利用$Minkowski$不等式
  \begin{equation*}
  \begin{split}
     \sqrt{\sum_{k=1}^{\infty}|\alpha_k|^2}&=\sqrt{\sum_{k=1}^{\infty}(|\alpha_k|-|\alpha_{n_0,k}|+|\alpha_{n_0,k}|)^2} \\
       &\leq\sqrt{\sum_{k=1}^{\infty}(|\alpha_k|-|\alpha_{n_0,k}|)^2}+\sqrt{\sum_{k=1}^{\infty}|\alpha_{n_0,k}|^2}\\
       &\leq\sqrt{\sum_{k=1}^{\infty}|\alpha_k-\alpha_{n_0,k}|^2}+\sqrt{\sum_{k=1}^{\infty}|\alpha_{n_0,k}|^2}<\infty
  \end{split}
  \end{equation*}
  这就证明了序列$a=(\alpha_n)$是$l^2$的元素. 因为$\varepsilon$是任意小的, 所以有
  \begin{equation*}
    \lim_{n\to\infty}\|a-a_n\|=\lim_{n\to\infty}\sqrt{\sum_{k=1}^{\infty}(|\alpha_k|-|\alpha_{n,k}|)^2}=0
  \end{equation*}
  序列$(a_n)$收敛于$a$,且$a\in l^2$
\end{example}
\subsubsection*{收敛与绝对收敛}
\begin{definition}
  赋范空间$E$中有级数$\sum_{n=1}^{\infty}x_n$, 如果序列的部分和收敛于$E$, 也就是说, 存在$x\in E$使得当$n\to\infty$时$\|x_1+x_2+\cdots+x_n-x\|\to0$成立, 则叫做收敛级数. 这种情况下记作$\sum_{n=1}^{\infty}x_n=x$. 如果$\sum_{n=1}^{\infty}\|x_n\|<\infty$, 则称级数为绝对收敛.
\end{definition}
\begin{theorem}
  Banach空间的闭子空间是Banach空间自己
\end{theorem}
\section{线性映射}
我们先介绍一些概念. 设$E_1$和$E_2$是两个向量空间, $L$是$E_1$到$E_2$的映射. 如果$y=L(x)$, 则称$y$是$x$的像.

\indent如果$A$是$E_1$的子集, 则$L(A)$表示$A$的像集, 也就是说, $L(A)$是$A$中元素的像的集合,并且像在$E_2$里. 如果$B$是$E_2$的子集, 则$L^{-1}(B)$表示$B$的原像.
\begin{equation*}
  L(A)={L(x):x\in A},\quad L^{-1}(B)={x\in E_1:L(x)\in B}
\end{equation*}
值得注意的是$L^{-1}$并不意味着$L$是可逆的.

\indent我们通常考虑的映射是定义在向量空间的真子空间上的. 所以提出$L$的定义域这个概念是很重要的, 我们用$\mathcal{D}(L)$表示$L$的定义域. 集合$L(\mathcal{D}(L))$叫做$L$的值域, 用$\mathcal{R}(L)$表示.
\begin{equation*}
  \mathcal{R}(L)=\{y\in E_2:L(x)=y,x\in\mathcal{D}(L)\}
\end{equation*}
$L$的零空间记作$\mathcal{N}(L)$, 这指的是所有的$x\in \mathcal{D}(L)$使得$L(x)=0$. 在最后简单介绍一下映射的图, 用$\mathcal{G}(L)$表示映射$L$的图,指的是如下定义的$E_1\times E_2$的子集
\begin{equation*}
  \mathcal{G}(L)=\{(x,y):x\in\mathcal{D}(L),y=L(x)\}
\end{equation*}
\subsection*{线性映射}
\begin{definition}
  如果映射$L:E_1\mapsto E_2$对于任意的$x,y\in E_1,\alpha,\beta\in \mathbf{F}$满足$L(\alpha x+\beta y)=\alpha L(x)+\beta L(y)$, 则称$L$为线性映射
\end{definition}
在向量空间中, 我们常用$Lx$的写法替代$L(x)$.
\subsection*{连续映射}
\begin{definition}
  设$E_1$和$E_2$是赋范空间, $F$是$E_1$到$E_2$的映射, 如果对于所有$E_1$中的序列$(x_n)$都收敛于$x_0$, 有序列$(F(x_n))$收敛于$F(x_0)$, 即如果$\|x_n-x_0\|\to0$意味着$\|F(x_n)-F(x_0)\|\to0$, 那么就说$F$在$x_0$连续. 如果$F$在任意$x\in E_1$都连续, 则称$F$是连续的.
\end{definition}
我们将会在量子力学里面遇到很多连续性问题.
\begin{example}
  赋范空间$E$的范数是$E\mapsto\mathbb{R}$的连续映射.
  \begin{proof}
    如果有$\|x_n-x\|\to0$,则$|\|x_n\|-\|x\||\leq\|x_n-x\|\to 0$
  \end{proof}
\end{example}
\begin{theorem}
  设$F:E_1\mapsto E_2$. 下列条件相互等价.
  \begin{enumerate}[(a)]
    \item $F$是连续的
    \item $E_2$的任意开子集$U$的原象$F^{-1}(U)$在$E_1$中仍然是开的.
    \item $E_2$的任意闭子集$S$的原象$F^{-1}(S)$在$E_1$中仍然是闭的.
  \end{enumerate}
\end{theorem}
\begin{theorem}\label{continuous linear map}
  如果存在$x_0\in E_1$, 有线性映射$L:E_1\mapsto E_2$在$x_0$连续, 则线性映射$L$是连续映射的.
\end{theorem}
\begin{proof}
  假设$L$在$x_0\in E_1$连续. 设$x$是$E_1$的任意一个元素$(x_n)$是收敛于$x$的序列. 自然地有序列$(x_n-x+x_0)$收敛于$x_0$,这样我们有$\|Lx_n-Lx\|=\|L(x_n-x+x_0)-Lx_0\|\to 0$. 得证
\end{proof}
\subsection*{有界线性映射}
\begin{definition}
  线性映射$L:E_1\mapsto E_2$,如果存在数$\alpha>0$使得对于所有的$x\in E_1$都有$\|Lx\|\leq\|x\|$,则$L$叫做有界线性映射.
\end{definition}
值得注意的是, 这个定义相当于说$L$在$E_1$的单位球面上以$\alpha$为界, 也就是说对于所有的$x\in E_1$有$\|Lx\|\leq\alpha$使得$\|x\|=1$
\begin{theorem}
  线性映射是连续的, 当且仅当它是有界的.
\end{theorem}
\begin{proof}
  如果$L$是有界线性映射,且有$\|x_n\|\to 0$, 则$\|Lx_n\|\leq\alpha\|x_n\|\to 0$. 即线性映射$L$在$0$连续, 根据定理\ref{continuous linear map}, $L$是连续映射.
  \indent如果$L$不是有界的, 那么对于任意$n\in N$, 存在$x_n\in E_1$使得$\|Lx_n\|>n\|x_n\|$.定义序列:
  \begin{equation*}
    y_n=\frac{x_n}{n\|x_n\|}
  \end{equation*}
  显然$y_n\to 0$,而对于任意$n\in N$有$\|Ly_n\|>1$, 所以$L$不是连续映射
\end{proof}
有限维空间的线性映射是有界的,上述定理也意味着对于线性映射连续和一致连续是等价的.

\indent如果加法和数乘如下定义, 向量空间$E_1$到$E_2$的所有线性映射也构成向量空间:
\begin{equation*}
  (L_1+L_2)x=L_1x+L_2x,\quad (\lambda L)x=\lambda(Lx)
\end{equation*}
如果$E_1$和$E_2$是赋范空间, 则所有$E_1\mapsto E_2$的有界线性映射的集合是上面定义的空间的线性子空间, 记作$\mathcal{B}(E_1,E_2)$
\begin{theorem}
  如果$E_1$和$E_2$是赋范空间, 则$\mathcal{B}(E_1,E_2)$也是赋范空间, 范数如下定义:
  \begin{equation*}\label{B norm}
    \|L\|=\sup_{\|x\|=1}\|Lx\|
  \end{equation*}
\end{theorem}
\begin{proof}
  如果$L_1,L_2\in\mathcal{B}(E_1,E_2)$有$x\in E_1$使得$\|x\|=1$, 则有
  \begin{equation*}
    \|L_1x+L_2x\|\leq\|L_1x\|+\|L_2x\|
  \end{equation*}
  这意味着
  \begin{equation*}
    \|L_1x+L_2x\|\leq\sup_{\|x\|=1}\|L_1x\|+\sup_{\|x\|=1}\|L_2x\|=\|L_1\|+\|L_2\|
  \end{equation*}
  因此有,
  \begin{equation*}
    \|L_1+L_2\|=\sup_{\|x\|=1}\|L_1x+L_2x\|\leq\|L_1\|+\|L_2\|
  \end{equation*}
  所以上述范数定义的确满足三角不等式.
\end{proof}
\eqref{B norm}定义的范数是$\mathcal{B}(E_1,E_2)$的标准范数, 当我们谈论起$\mathcal{B}(E_1,E_2)$的范数时, 指的就是这个范数.
\begin{theorem}
  若$E_1$是赋范空间$E_2$是Banach空间, 则$\mathcal{B}(E_1,E_2)$是Banach空间
\end{theorem}
\begin{proof}
  首先说明$\mathcal{B}(E_1,E_2)$是完备的. $(L_n)$是$\mathcal{B}(E_1,E_2)$中的柯西序列, $x$是$E_1$的任意元素.则有:
  \begin{equation*}
    \|L_mx-L_nx\|\leq\|L_m-L_n\|\|x\|\to 0\quad m,n\to\infty
  \end{equation*}
  这表明了$(L_nx)$是$E_2$中的Cauchy序列. 由于$E_2$的完备性, 存在唯一的元素$y\in E_2$使得$L_nx\to y$. 因为$x$是$E_1$的任意元素, 这定义了一个从$E_1$到$E_2$的映射$L$:
  \begin{equation*}
    Lx=\lim_{n\to\infty}L_nx
  \end{equation*}
  接下来再证明$L\in\mathcal{B}(E_1,E_2)$和$\|L_n-L\|\to 0$.
  显然, $L$是一个线性映射. 因为Cauchy序列有界, 存在一个常数$\alpha$使得对于所有的$n\in \mathbb{N}$有$\|L_n\|\leq\alpha$成立. 所以
  \begin{equation*}
    \|Lx\|=\|\lim_{n\to\infty}L_nx\|=\lim_{n\to\infty}\|L_nx\|\leq\alpha\|x\|
  \end{equation*}
  因此$L$也是有界的, 这样$L\in\mathcal{B}(E_1,E_2)$. 接下来还有$\|L_n-L\|\to0$需要证明. 令$\varepsilon>0$, 存在$k$使得对于所有的$m,n\geq k$都有$\|L_m-L_n\|<\varepsilon$. 如果$\|x\|=1,m,n\geq k$,则
  \begin{equation*}
    \|L_mx-L_nx\|\leq\|L_m-L_n\|<\varepsilon
  \end{equation*}
  令$n\to\infty$得到对于任意$m\geq k,x\in E_1,\|x\|=1$有$\|L_mx-Lx\|\leq\varepsilon$. 所以对于所有的$m>k$, $\|L_m-L\|\leq\varepsilon$.得证
\end{proof}
\begin{theorem}
  (对角定理)$E$是一个赋范空间,$(x_{ij}),(i,j\in \mathbb{N})$是$E$中的无限维矩阵的矩阵元. 若有:
  \begin{enumerate}[(a)]
    \item 对于任意$j\in \mathbb{N}$有$\lim_{i\to\infty}x_{ij}=0$
    \item 对于每一个指标列$p_n$有子列$q_n$使得
    \begin{equation*}
      \lim_{i\to\infty}\sum_{j=1}^{\infty}x_{q_jq_j}=0
    \end{equation*}
    则$\lim_{i\to\infty}x_{ii}=0$
  \end{enumerate}
\end{theorem}
留作练习, 读者自证. 下面简单介绍一下压缩映射和不动点定理
\subsection*{压缩映射}
\begin{definition}
  赋范空间$E$的子集$A$到$E$之间有一映射$f$, 如果存在正数$\alpha<1$使得对于所有的$x,y\in A$都有:
  \begin{equation*}
    \|f(x)-f(y)\|\leq\alpha\|x-y\|
  \end{equation*}
  则把$f$叫做压缩映射.
\end{definition}
\begin{example}
  考虑一个非线性方程$x^3-x-1=0$. 方程有三个根. 把方程写成$Tx=x$形式有多种方式.例如
  \begin{equation*}
    Tx=(1+x)^{\frac{1}{3}},\quad Tx=x^3-1,\quad Tx=\frac{1}{x^2-1}
  \end{equation*}
  原方程的根在$[1,2]$. 由$T(x)=(1+x)^{\frac{1}{3}}$定义的映射$T$是一个压缩映射. 由均值定理可以得到
  \begin{equation*}
    |Tx-Ty|=|(1+x)^{\frac{1}{3}}-(1+y)^{\frac{1}{3}}|\leq\frac{2^{\frac{1}{3}}}{6}|x-y|
  \end{equation*}
  其余两个不是压缩映射
\end{example}
\subsection*{Banach不动点定理}
首先要介绍一下不动点. 如果映射$f$满足存在点$z$使得$f(z)=z$, 则称$z$为$f$的不动点. 这个定理在后面的微分积分方程里有很大的用处.
\begin{example}
  令$E=\mathcal{C}([0,1])$是定义在闭区间$[0,1]$上的连续复函数空间. 再令$T$如下定义
  \begin{equation*}
    (Tx)(t)=x(0)+\int_{0}^{t}x(\tau)d\tau
  \end{equation*}
  显然, 对于任意$a\in \mathbb{C}$, 函数$x(t)=ae^t$是$T$的不动点.
\end{example}
\begin{theorem}
  (不动点定理)令$F$是赋范空间$E$的闭子集, $f$是从$F$到$F$的压缩映射, 则存在唯一的$x\in F$使得$f(z)=z$
\end{theorem}
\begin{proof}
  令$0<\alpha<1$使得对于所有的$x,y\in F$有
  \begin{equation*}
    \|f(x)-f(y)\|\leq\alpha\|x-y\|
  \end{equation*}
  令$x_0$是$F$中的任意点, $x_n=f(x_{n-1}),n=1,2,\dots$. 我们将证明$(x_n)$是Cauchy序列. 首先观察到, 对于任意$n\in \mathbb{N}$
  \begin{equation*}
    \|x_{n+1}-x_n\|\leq\alpha\|x_n-x_{n-1}\|\leq\alpha^2\|x_{n-1}-x_{n-2}\|\leq\cdots\leq\alpha^n\|x_1-x_0\|
  \end{equation*}
  因此对于任意$m,n\in\mathbb{N}$使得$m<n$有
  \begin{equation*}
  \begin{split}
     \|x_n-x_m\| & \leq\|x_n-x_{n-1}\|+\|x_{n-1}-x_{n-2}\|+\cdots+\|x_{m+1}-x_{m}\| \\
       & \leq(\alpha^{n-1}+\alpha^{n-2}+\cdots+\alpha^m)\|x_1-x_0\| \\
       & \frac{\|x_1-x_0\|}{1-\alpha}\alpha^m\to 0,\quad m\to\infty
  \end{split}
  \end{equation*}
  这样$(x_n)$是Cauchy序列. 因为$F$是完备空间的闭子集, 所以存在$z\in F$使得当$n\to\infty$时$x_n\to z$.接下来证明唯一存在$z$使得$f(z)=z$.
  首先因为
  \begin{equation*}\
  \begin{split}
     \|f(z)-z\|&\leq\|f(z)-x_n\|+\|x_n-z\| \\
       &=\|f(z)-f(x_{n-1})\|+\|x_n-z\| \\
       &\leq\alpha\|x-z_{n-1}\|+\|x_n-z\|\to 0,\quad n\to\infty
  \end{split}
  \end{equation*}
  $\|f(z)-z\|=0$, 这样就有$f(z)=z$. 假设有另一个点$w\in F$,$f(w)=w$, 则
  \begin{equation*}
    \|z-w\|=\|f(z)-f(w)\|\leq\alpha\|z-w\|
  \end{equation*}
  因为$0\leq\alpha\leq1$,必有$\|z-w\|=0$,这意味着$z=w$
\end{proof}
\section{线性空间的同构}
\indent设$a_1,a_2,\cdots,a_n$是线性空间$E$的一组基, 在这组基下, $E$中每一个向量都有确定的坐标, 而向量的坐标可以看成$\mathbf{F}^n$的元素.因此向量与它的坐标之间的对应实质上就是$E$到$\mathbf{F}$的一个映射, 这个映射是单射且满射的. 坐标给出了线性空间$E$与$\mathbf{F}^n$的双射.这个对应最重要体现在运算关系的对应上.

\indent设
\begin{equation*}
\begin{split}
   \mathbf{x}&=x_1a_1+x_2a_2+\cdots+x_na_n=\sum_{i=1}^{n}x_ia_i \\
   \mathbf{y}&=y_1a_1+y_2a_2+\cdots+y_na_n=\sum_{j=1}^{n}y_ja_j
\end{split}
\end{equation*}
即向量$\mathbf{x},\mathbf{y}$的坐标分别是$(x_1,\cdots,x_n),(y_1,\cdots,y_n)$, 那么
\begin{equation*}
\begin{split}
  \mathbf{x}+\mathbf{y}&=(x_1+y_1)a_1+\cdots+(x_n+y_n)a_n\\
  k\mathbf{x}&=kx_1a_1+\cdots+kx_na_n
\end{split}
\end{equation*}
向量$\mathbf{x}+\mathbf{y},k\mathbf{x}$的坐标分别是
\begin{equation*}
\begin{split}
  &(x_1+y_1,x_2+y_2,\cdots,x_n+y_n)=(x_1,\cdots,x_n)+(y_1,\cdots,y_n)\\
  &(kx_1,kx_2,\cdots,kx_n)=k(x_1,x_2,\cdots,x_n)
\end{split}
\end{equation*}
向量的运算可以归结成它们坐标的运算, 因而线性空间$E$的讨论可以归结于$\mathbf{F}^n$的讨论
\begin{definition}
  数域$\mathbf{F}$上两个线性空间$E$与$E'$称为同构的, 如果从$E$到$E'$有一个双射$\sigma$. 具有以下性质:
  \begin{enumerate}[(1)]
    \item $\sigma(a+b)=\sigma(a)+\sigma(b)$
    \item $\sigma(ka)=k\sigma(a)$
  \end{enumerate}
  \indent其中$a,b$是$E$中任意向量, k是$\mathbb{F}$中任意数, 这样的映射称为同构映射.
\end{definition}
\indent同构映射基本性质:
\begin{enumerate}[(1)]
  \item $\sigma(0)=0,\sigma(-a)=-\sigma(a)$
  \item $\sigma(\sum_{i=1}^{n}k_ia_i)=\sum_{i=1}^{n}k_i\sigma(a_i)$
  \item $E$中向量组$a_1,\cdots,a_n$线性相关的充分必要条件是, 它们的像$\sigma(a_1),\sigma(a_2),\cdots,\sigma(a_n)$线性相关
  \item 如果$E_1$是$E$的一个线性子空间,那么, $E_1$在$\sigma$下像集合
  \begin{equation*}
    \sigma(E_1)={\sigma(a)|a\in E_1}
  \end{equation*}
  是$\sigma(E)$的子空间, 并且$dim(E_1)=dim(\sigma(E_1))$
  \item 同构映射的逆以及两个同构映射的乘积还是同构映射
\end{enumerate}
\section{线性变换}
\subsection*{线性变换基本概念}
\indent线性空间中,事物之间的联系反映为线性空间的映射,线性空间到自身的映射通常称为$E$的一个变换.线性变换是最简单最基本的一种变换,线性变换是量子力学里最常用的概念.
\begin{definition}
  如果对于线性空间$E$中任意元素$a,b$和数域$\mathbf{F}$中任意数$k$, 都有这样的线性变换$A$, 满足:
  \begin{equation*}
  \begin{split}
     A(a+b)=A(a)+A(b)\\
     &A(ka)=kA(a)
  \end{split}
  \end{equation*}
  则这个变换$A$称为线性变换, 以后使用大写$A,B,\cdots$代表$E$的变换,$A(a)$或$Aa$代表元素$a$在变换$A$下的像.\textbf{线性变换保持加法与数量乘法}.
\end{definition}
\begin{example}
  平面上的向量构成实数域上的二维线性空间.把平面围绕坐标原点绕逆时针方向转动$\theta$角, 就是一个线性变换, 用$A_\theta$表示.如果平面上一个向量$a$在直角坐标系下的坐标是$(x,y)$, 那么像$A_\theta(a)$的坐标(x',y')为:
  \begin{equation*}
    \begin{pmatrix}
      x' \\
      y' \\
    \end{pmatrix}=\begin{pmatrix}
                    cos\theta & -sin\theta \\
                    sin\theta & cos\theta \\
                  \end{pmatrix}\begin{pmatrix}
                                 x \\
                                 y \\
                               \end{pmatrix}
  \end{equation*}
  \indent同样的三维空间中绕轴有限转动也是一个线性变换
\end{example}
\begin{example}
  线性空间$E$中的恒等变换(单位变换)满足:
  \begin{equation*}
    I(a)=a\quad(a\in E)
  \end{equation*}
  以及零变换$0$:$0(a)=0\quad (a\in E)$
\end{example}
\subsection*{线性变换的运算}
线性变换作为映射, 利用映射的复合可以定义线性变换的乘法
\begin{definition}
  设$A,B$是线性空间$E$上的两个线性变换,定义他们乘积$AB$为
  \begin{equation*}
    (AB)(a)=A(B(a))\quad (a\in E)
  \end{equation*}
  容易得到, 线性变换的乘积也是线性变换(读者自证)
\end{definition}
通过以上定义可以看出线性变换满足乘法结合律
\begin{equation*}
  (AB)C=A(BC)
\end{equation*}
但线性变换的乘法一般是不可交换的.
\begin{example}
  实数域$R$上的线性空间$R[x]$中, 线性变换
  \begin{equation*}
    \begin{split}
       D(f(x))&=f'(x)\\
       F(f(x))&=\int_{0}^{x}f(t)dt
    \end{split}
  \end{equation*}
  的乘积$DF=I$,但一般$FD\neq I$
\end{example}
线性变换还可以定义加法.设$A,B$是线性空间$E$上的两个线性变换, 定义它们的和$A+B$为
\begin{equation*}
  (A+B)(a)=A(a)+B(a)\quad a\in E
\end{equation*}
线性变换的和还是线性变换, 不难证明线性变换的加法满足加法结合律与交换律. 对于加法, 零变换$0$有特殊地位, 它与所有线性变换$A$的和仍为$A$:
\begin{equation*}
  A+0=A
\end{equation*}
同样可以定义负变换
\begin{equation*}
  (-A)(a)=-A(a)\quad a\in E
\end{equation*}
显然:$A+(-A)=0$.

\section{Hilbert空间}
在复习Hilbert空间之前,我们先复习对偶空间的概念.设$E$是实数域$\mathbb{R}$上的$n$维向量空间, 可定义$E$上的线性映射$L$. 即向量空间$E$与实数域$\mathbb{R}$间的线性映射
\begin{equation*}
\begin{split}
   L&:R\mapsto R \\
     x\mapsto&L(x)\in \mathbb{R},\quad x\in E
\end{split}
\end{equation*}
线性映射保持线性结构.
\begin{equation*}
  L(\alpha x+\beta y)=\alpha L(x)+\beta L(y),\quad x,y\in E,\alpha,\beta\in\mathbb{R}
\end{equation*}
在之前的线性映射章节里, 我们讨论了$E_1\mapsto E_2$的线性映射的集合也构成向量空间$\mathcal{B}(E_1,E_2)$.这里我们把这个向量空间$\mathcal{B}(E,\mathbb{R})$称为向量空间$E$的对偶空间, 记作$E^*$
\begin{equation*}
  L\in E^*=\mathcal{B}(E,\mathbb{R})
\end{equation*}
如果$(e_1,e_2,\dots,e_n)$为向量空间的$E$一组基底, 任意向量$x$可用这组基展开
\begin{equation*}
  x=x_1e_1+x_2e_2+\cdots+x_ne_n, \quad x_i\in\mathbb{R},n=1,2,\dots
\end{equation*}
由于线性映射保证线性结构不变.
\begin{equation*}
  Lx=x_1Le_1+x_2Le_2+\cdots+x_nLe_n, \quad x_i\in\mathbb{R},n=1,2,\dots
\end{equation*}
即所有的线性映射$L$都可以由它们在基底$e_i$上取的值决定.记
\begin{equation*}
  L_i=Le_i\in\mathbb{R}
\end{equation*}
称为对偶向量$L$的分量, 可将$L\in E^*$记作
\begin{equation*}
  f=f_1e_1^*+f_2e_2^*+\cdots f_ne_n^*
\end{equation*}
其中$e_i^*\in E^*$, 为向量空间中满足下列条件的线性映射.
\begin{equation*}
  e_i^*(e_k)=\delta_{ik}
\end{equation*}
显然这组线性映射$e_1^,e_2^*,\dots,e_n^*$线性无关. 它们形成了$E^*$中的一组基, 称为与基$(e_1,e_2,\dots,e_n)$对偶的基. 对偶空间$E^*$也为$n$维向量空间, 且空间$E$与空间$E^*$相互对偶.

\indent向量空间$E$和$E$上的线性映射空间$E^*$是两个重要概念, 为帮助熟悉他的含义, 我们采用熟悉的Dirac符号去表示.向量空间$E$上的元素$x$可以用$|x\rangle$去表示. 基组$(e_1,e_2,\dots,e_n)$可以用$(|1\rangle,|2\rangle,\dots,|n\rangle)$去表示.
\begin{equation*}
  |x\rangle=x_1|1\rangle+x_2|2\rangle+\cdots+x_n|n\rangle
\end{equation*}
对偶空间$E^*$上的元素$L$记作$\langle L|$, 基组$(e_1^*,e_2^*,\dots,e_n^*)$记作$(\langle1|,\langle2|,\dots,\langle n|)$
\begin{equation*}
  \langle L|=L_1\langle1|+L_2\langle2|+\cdots+L_n\langle n|
\end{equation*}
基矢$\langle i|,i=1,\dots,n$由它对整个向量空间$E$的作用决定
\begin{equation*}
  \langle i|j\rangle=\delta_{ij}
\end{equation*}
值得注意, $E$与$E^*$均为线性空间, 在$E$与$E^*$之间可以定义内积
\begin{equation*}
  \langle L|x\rangle=\sum_{k=1}^{k=n}L_kx_k
\end{equation*}
\subsection*{内积空间}
我们现在开始讨论定义了内积的复向量空间, 内积的定义如下. 这里$\mathbf{F}=\mathbb{C}$, 对于$x\in\mathbb{C}$, 复共轭的标记我们用$x^*$表示.
\begin{definition}\label{inner def}
  令$E$是一个复向量空间, 有映射$\langle\cdot,\cdot\rangle\mapsto\mathbb{C}$, 如果对于任意$x,y,z\in E$和$\alpha,\beta\in \mathbb{C}$满足下列条件
  \begin{enumerate}[(a)]
    \item\label{inner a} $\langle x,y\rangle=\langle y,x\rangle^*$
    \item\label{inner b} $\langle x,\alpha y+\beta z\rangle=\alpha\langle x,z\rangle+\beta\langle y,z\rangle$
    \item $\langle x,x\rangle\geq 0$
    \item $\langle x,x\rangle=0$当且仅当$x=0$
  \end{enumerate}
  则把这个映射叫做内积.具有内积的向量空间叫做向量空间
\end{definition}
根据定义, 两个向量的内积是复数. 根据\ref{inner a}, $\langle x,x\rangle=\langle x,x\rangle^*$, 对于任意$x\in E$$\langle x,x\rangle$是实数. 根据\ref{inner b}可以得到关系式
\begin{equation*}
  \langle\alpha x+\beta y,z\rangle=\langle z,\alpha x+\beta y\rangle^*=(\alpha\langle z,x\rangle+\beta\langle z,y\rangle)^*=\alpha^*\langle z,x\rangle^*+\beta^*\langle z,y\rangle^*=\alpha^*\langle x,z\rangle+\beta^*\langle y,z\rangle
\end{equation*}
特别的
\begin{equation*}
  \langle\alpha x,y\rangle=\alpha^*\langle x,y\rangle,\quad \langle x,\alpha y\rangle=\alpha\langle x,y\rangle
\end{equation*}
因此如果$\alpha=0$我们有$\langle 0,y\rangle=\langle x,0\rangle=0$
\begin{example}
  最简单最重要的内积空间例子是复数空间$\mathbb{C}$. 其中内积的定义是$\langle x,y\rangle=x^*y$
\end{example}
\begin{example}
  $n$元复数数组$x=(x_1,\dots,x_n)$的空间$\mathbb{C}^n$, 构成内积空间, 其内积的定义是
  \begin{equation*}
    \langle x,y\rangle=\sum_{k=1}^{n}x_k^* y_k,\quad x=(x_1,\dots,x_n),y=(y_1,\dots,y_n)
  \end{equation*}
\end{example}
\begin{example}\label{l^2 inner space}
  所有复数序列$(x_1,x_2,x_3,\dots)$构成的$l^2$空间$(\sum_{k=1}^{\infty}|x_k|^2<\infty)$是一个内积空间, 其内积定义是
  \begin{equation*}
    \langle x,y\rangle=\sum_{k=1}^{\infty}x_k^*y_k,\quad x=(x_1,x_2,x_3,\dots),y=(y_1,y_2,y_3\dots)
  \end{equation*}
  之后我们将会见到, 这个空间是内积空间里最重要的例子.
\end{example}
\begin{example}\label{nonzero term}
  仅有限非零项的复数序列$(x_1,x_2,x_3,\dots)$构成的空间是一个内积空间, 内积的定义如\ref{l^2 inner space}.
\end{example}
\begin{example}\label{C[a,b]inner}
  区间$[a,b]$上的连续复函数的空间$\mathcal{C}([a,b])$是一个内积空间, 内积定义如下
  \begin{equation*}
    \langle f,g\rangle=\int_{a}^{b}f^*(x)g(x)dx
  \end{equation*}
\end{example}
\begin{example}
  $L^2(\mathbb{R})$是内积空间, 内积定义如下:
  \begin{equation*}
    \int_{-\infty}^{+\infty}f^*(x)g(x)dx
  \end{equation*}
  进一步说, $L^2(\mathbb{R}^n)$也是内积空间, 其内积定义是
  \begin{equation*}
    \int_{\mathbb{R}^n}f^*(x)g(x)dx
  \end{equation*}
\end{example}
\begin{example}
  令$E$是内积空间$E_1$和$E_2$的Cartesian乘积, $E=E_1\times E_2={(x,y):x\in E_1,y\in E_2}$. 空间$E$是一个内积空间, 其内积定义如下:
  \begin{equation*}
    \langle(x_1,y_1),(x_2,y_2)\rangle=\langle x_1,x_2\rangle+\langle y_1,y_2\rangle
  \end{equation*}
  值得注意$E_1$和$E_2$可以分别的看成子空间$E\times{0}$和${0}\times E_2$
\end{example}
内积空间是具有内积的向量空间, 每一个内积空间也是赋范空间, 范数如下定义
\begin{equation*}
  \|x\|=\sqrt{\langle x,x\rangle}
\end{equation*}
首先注意函数是良定义的, 因为$\rangle x,x\rangle$总是非负的实数, 并且$\|x\|=0$当且仅当$x=0$. 进一步说
\begin{equation*}
  \|\lambda x\|=\sqrt{\langle\lambda x,\lambda x\rangle}=\sqrt{\lambda^*\lambda\langle x,x\rangle}=|\lambda|\|x\|.
\end{equation*}
还剩下三角不等式有待证明, 这不像前两个条件这样好证明. 首先我们介绍一下Cauchy-Schwarz不等式.
\begin{theorem}
  对于内积空间里的任意两个元素$x,y$, 有
  \begin{equation}\label{Cauchy-Schwarz}
    |\langle x,y\rangle|\leq\|x\|\|y\|
  \end{equation}
  当且仅当$x,y$线性相关时, 等式$|\langle x,y\rangle|=\|x\|\|y\|$成立.
\end{theorem}
\begin{proof}
  若$y=0$, 则\eqref{Cauchy-Schwarz}刚好满足, 左右两边均是$0$. 假设$y\neq 0$, 我们有
  \begin{equation*}\label{Cauchy-Schwarz proof1}
    0\leq\langle x+\alpha y,x+_\alpha y\rangle=\langle x,x\rangle+\alpha\langle x,y\rangle+\alpha^*\langle y,x\rangle+|\alpha|^2\langle y,y\rangle
  \end{equation*}
  在\eqref{Cauchy-Schwarz proof1}中令$\alpha=-\frac{\langle y,x\rangle}{\langle y,y\rangle}$. 两边同时乘以$\langle y,y\rangle$.
  \begin{equation*}
    0\leq\langle x,x\rangle\rangle y,y\rangle-|\langle x,y\rangle|^2
  \end{equation*}
  这就给出了Cauchy-Schwarz不等式. 如果$x,y$线性相关, 存在$\alpha\in \mathbb{C}$有$x=\alpha y$. 因此
  \begin{equation*}
    |\langle x,y\rangle|=|\langle x,\alpha x\rangle|=|\alpha|\langle x,x\rangle=|\alpha|\|x\|\|x\|=\|x\|\|\alpha x\|=\|x\|\|\alpha y\|
  \end{equation*}
  令$x,y$是两个向量,使得$|\langle x,y\rangle|=\|x\|\|y\|$, 也就是说
  \begin{equation*}
    \langle x,y\rangle\langle y,x\rangle=\langle x,x\rangle\langle y,y\rangle
  \end{equation*}
  接下来利用Cauchy-Schwarz不等式等号成立条件去证明$\langle y,y\rangle x-\langle y,x\rangle y=0$, 这表明了$x,y$线性相关.
  \begin{equation*}
  \begin{split}
     \langle\langle y&,y\rangle x-\langle y,x\rangle y,\langle y,y\rangle x-\langle y,x\rangle y\rangle \\
       & =|\langle y,y\rangle|^2\langle x,x\rangle-\langle y,y\rangle\langle y,x\rangle\langle x,y\rangle-\langle y,x\rangle\langle y,y\rangle\langle y,x\rangle+|\langle x,y\rangle|\langle y,x\rangle\langle y,y\rangle \\
       & =0
  \end{split}
  \end{equation*}
  证毕.
\end{proof}
\begin{corollary}
  (三角不等式)对于内积空间中的任意两个元素$x,y$有:
  \begin{equation*}
    \|x+y\|\leq\|x\|+\|y\|
  \end{equation*}
\end{corollary}
\begin{proof}
  \begin{equation*}
  \begin{split}
     \|x+y\|^2& =\langle x+y,x+y\rangle=\langle x,x\rangle+2Re\langle x,y\rangle+\langle y,y\rangle \\
       & \leq\langle x,x\rangle+2|\langle x,y\rangle|+\langle y,y\rangle \\
       & \leq\|x\|^2+2\|x\|\|y\|+\|y\|^2 \\
       & (\|x\|+\|y\|)^2
  \end{split}
  \end{equation*}
  其中$Re z$表示$z\in \mathbb{C}$的实部.
\end{proof}
\subsubsection*{内积空间的范数}
\begin{definition}
  根据$\|x\|=\sqrt{\langle x,x\rangle}$可以定义内积空间$E$的范数.
\end{definition}
我们已经证明了每一个内积空间都是赋范空间. 很自然的我们会问到, 是否每一个赋范空间是内积空间呢? 更精确地表述是:是否可能在赋范空间$(E,\cdot)$中定义一个内积$\langle\cdot,\cdot\rangle$使得对于每一个$x\in E$, 有 $\|x\|=\sqrt{\langle x,x\rangle}$? 一般来说, 答案是否定的. 接下来我们将会给出内积空间范数的性质, 这些性质对于赋范空间称为内积空间是充分必要的.
\begin{theorem}
  (平行四边形法则)对于内积空间中任意两个元素, 有:
  \begin{equation*}
    \|x+y\|^2+\|x-y\|^2=2(\|x\|^2+\|y\|^2)
  \end{equation*}
\end{theorem}
\begin{proof}
  我们有
  \begin{equation*}
    \|x+y\|^2=\langle x+y,x+y\rangle=\langle x,x\rangle+\langle x,y\rangle+\langle y,x\rangle+\langle y,y\rangle
  \end{equation*}
  因此
  \begin{equation*}
    \|x+y\|^2=\|x\|^2+\langle x,y\rangle+\langle y,x\rangle+\|y\|^2
  \end{equation*}
  现在令$y\to -y$, 上式得到
  \begin{equation*}
    \|x-y\|^2=\|x\|^2-\langle x,y\rangle-\langle y,x\rangle+\|y\|^2
  \end{equation*}
  两式叠加可得平行四边形法则
\end{proof}
内积空间里最重要的是可以定义正交向量.这使得Hilbert空间与Banach空间非常的不同.
\begin{definition}
  (正交向量)内积空间里有两个向量$x,y$, 如果$\langle x,y\rangle=0$, 则称这两个向量正交,记作$x\perp y$
\end{definition}
如果$x\perp y$, 则有$\langle y,x\rangle=\langle x,y\rangle^*=0$, 因此$y\perp x$. 换句话说, $\perp$关系式对称的.
\begin{theorem}
  (勾股定理)对于任意一对正交向量$x,y$, 我们有
  \begin{equation*}
    \|x+y\|^2=\|x\|^2+\|y\|^2
  \end{equation*}
\end{theorem}
\begin{proof}
  如果$x\perp y$, 则$\langle x,y\rangle=0$, 根据平行四边形法则, 立即可以得到上式.
\end{proof}
在之前我们定义的内积空间都是在$\mathbb{C}$中定义的. 我们也可以在$\mathbb{R}$上定义内积空间, 也就是说任意两个向量的内积是一个实数. 内积的定义\ref{inner def}中的\ref{inner b}要修改成$\langle x,y\rangle=\langle y,x\rangle$, 而之前所有的定理和例子都要修改成$\mathbb{R}$上成立. 有限维实内积空间也叫做Euclidean空间

\indent如果$R^n$上有$x=(x_1,x_2,\dots),y=(y_1,y_2,\dots)$, 则内积的定义$\langle x,y\rangle=\sum_{k=1}^{n}x_ky_k$等价于定义$\langle x,y\rangle=\|x\|\|y\|cos\theta$, 其中$\theta$是向量$x,y$的夹角. 在这种情况下, Cauchy-Schwarz不等式成为
\begin{equation*}
  \frac{|\langle x,y\rangle|}{\|x\|\|y\|}=cos\theta\leq 1,\quad x\neq0,y\neq0
\end{equation*}
\subsection*{希尔伯特空间}
\begin{definition}
  完备内积空间叫做Hilbert空间
\end{definition}
这里$E$的完备性是指用$E$的内积定义的范数的完备性.我们现在给出一些内积空间和Hilbert空间的例子
\begin{example}
  由于$\mathbb{C}$与$\mathbb{C}^n$是完备的, 它们都是Hilbert空间.
\end{example}
\begin{example}
  $l^2$是Hilbert空间, 其完备性在之前例子\ref{l^2 complete}已经证明过了, 这里不再复述.
\end{example}
\begin{example}
  例子\ref{nonzero term}里的空间$E$不是Hilbert空间, 考虑柯西序列序列
  \begin{equation*}
    x_n=(1,\frac{1}{2},\frac{1}{3},\dots,\frac{1}{n},0,0,\dots)
  \end{equation*}
  序列满足
  \begin{equation*}
    \lim_{m,n\to\infty}\|x_m-x_m\|=\lim_{n,m\to\infty}\left[\sum_{k=\min{\{m,n\}+1}}^{\max\{m,n\}}\frac{1}{k^2}\right]^{\frac{1}{2}}=0
  \end{equation*}
  然而这个序列并不收敛于$E$中. 因为极限$(1,\frac{1}{2},\frac{1}{3},\dots)$不在$E$中.
\end{example}
\begin{example}
  例子\ref{C[a,b]inner}中的空间是另一个不完备的内积空间例子. 事实上,考虑$\mathcal{C}([0,1])$中的如下函数序列:
  \begin{equation*}
    f_n{x}=\begin{cases}
             1  & \text{如果$0\leq x\leq\frac{1}{2}$} \\
             1-2n\left(x-\frac{1}{2}\right)  & \text{如果$\frac{1}{2}\leq x\leq \frac{1}{2n}+\frac{1}{2}$} \\
             0  & \text{如果$\frac{1}{2n}+\frac{1}{2}\leq x\leq 1$}
           \end{cases}
  \end{equation*}
  显然$f_n$是连续的. 进一步可知
  \begin{equation*}
    \|f_n-f_m\|\leq\left(\frac{1}{n}+\frac{1}{m}\right)^\frac{1}{2}\to 0,\quad\text{当$m,n\to\infty$}
  \end{equation*}
  因此$(f_n)$是柯西序列. 显然这个序列是逐点收敛于函数
  \begin{equation*}
    f(x)=\begin{cases}
           1 & \text{如果$0\leq x\leq \frac{1}{2}$}\\
           0 & \text{如果$\frac{1}{2}\leq x\leq 1$}
         \end{cases}
  \end{equation*}
  函数极限不连续, 因此它不属于$\mathcal{C}([0,1])$, 序列$(f_n)$不收敛于$\mathcal{C}([0,1])$. $\mathcal{C}([0,1])$不是Hilbert空间.
\end{example}
\begin{example}
  $L^2(R)$和$L^2([a,b])$是Hilbert空间
\end{example}
\begin{example}
  $\rho$是区间$[a,b]$上的可测函数且在$[a,b]$上$\rho(x)>0$.记$L^{2,\rho}([a,b])$为$[a,b]$上复可测函数构成的空间
  \begin{equation*}
    \int_{a}^{b}|f(x)|^2\rho(x)dx<\infty
  \end{equation*}
  这是定义了内积
  \begin{equation*}
    \langle f,g\rangle=\int_{a}^{b}f^*(x)g(x)\rho(x)dx
  \end{equation*}
  的一个Hilbert空间. 为了证明完备性, 考虑$L^{2,\rho}([a,b])$上的柯西序列$(f_n)$.
  \begin{equation*}
    \|f_m-f_n\|^2_{L^{2,\rho}([a,b])}=\int_{a}^{b}|f_m(x)-f_n(x)|^2\rho(x)dx\to 0\quad\text{当$n\to\infty$}
  \end{equation*}
  定义
  \begin{equation*}
    F_n=f_n\sqrt{\rho},\quad n\in\mathbb{N}
  \end{equation*}
  因为
  \begin{equation*}\
  \begin{split}
     \|F_m-F_n\|^2_{L^2{([a,b])}} & =\int_{a}^{b}|F_m(x)-F_n(x)|^2dx \\
       & =\int_{a}^{b}|f_m(x)\sqrt{\rho(x)}-f_n(x)\sqrt{\rho(x)}|^2dx\\
       & =\int_{a}^{b}|f_m(x)-f_n(x)|^2\rho(x)dx \\
       & =\|f_m-f_n\|^2_{L^{2,\rho}([a,b])}
  \end{split}
  \end{equation*}
  $(F_n)$是$L^2([a,b])$中的柯西序列.因此存在$F\in L^2([a,b])$使得
  \begin{equation*}
    \|F_n-F_m\|^2_{L^2([a,b])}=\int_{a}^{b}|F_n(x)-F(x)|^2dx\to 0
  \end{equation*}
  我们很容易看到$f_n\to\frac{F}{\sqrt{\rho}} \in L^{2,\rho}([a,b])$, 证明完毕
\end{example}
\begin{example}
  (Sobolev空间)$\Omega$是$\mathbb{R}^n$的一个开集. 所有复函数$f\in\mathcal{C}^m(\Omega)$构成的空间记作$\tilde{H}^m(\Omega),m=1,2,\dots$, 使得对于所有的$|\alpha|\leq m$有$D^\alpha f\in L^2(\Omega)$成立, 其中$\alpha=(\alpha_1,\dots,\alpha_n), \alpha_1,\dots,\alpha_n$是非负整数, $|\alpha|=\alpha_1+\cdots+\alpha_n$, 以及
  \begin{equation*}
    D^\alpha f=\frac{\partial^{|\alpha|f}}{\partial x_1^{\alpha_1}\partial x_2^{\alpha_2}\dots\partial x_n^{\alpha_n}}
  \end{equation*}
  例如, 如果$n=2,\alpha=(2,1)$, 我们有
  \begin{equation*}
    D^\alpha f=\frac{\partial^3 f}{\partial x_1^2\partial x_2}
  \end{equation*}
  对于$f\in\tilde{H}^m(\Omega)$, 对于每一个多元指标$\alpha=(\alpha_1,\alpha_2,\dots,\alpha_n)$, 我们有
  \begin{equation*}
    \int_{\Omega}|\frac{\partial^{|\alpha| f}}{\partial x_1^{\alpha_1}\partial x_2^{\alpha_2}\dots\partial x_n^{\alpha_n}}|^2<\infty
  \end{equation*}
  使得$|\alpha|\leq m$. $\tilde{H}^m(\Omega)$中的内积如下定义:
  \begin{equation*}
    \langle f,g\rangle=\int_{\Omega}\sum_{|\alpha|\leq m}(D^\alpha f)^*D^\alpha g
  \end{equation*}
  如果$\Omega\in \mathbb{R}^2$, $\tilde{H}^2(\Omega)$的内积就是
  \begin{equation*}
    \langle f,g\rangle=\int_{\Omega}(f^*g+f_x^*g_x+f_y^*g_y+f_{xx}^*g_{xx}+f_{yy}^*g_{yy}+f_{xy}^*g_{xy})
  \end{equation*}
  如果$\Omega=(a,b)\subset \mathbb{R}$, $\tilde{H}^m(a,b)$的内积就是
  \begin{equation*}
    \langle f,g\rangle=\int_{a}^{b}\sum_{n=0}^{m}(\frac{d^n f}{dx^n})^* \frac{d^n g}{dx^n}
  \end{equation*}
  $\tilde{H}^m(\Omega)$是一个内积空间, 但不是Hilbert空间, 因为它不完备. 完备的$\tilde{H}^m(\Omega)$记作$H^m(\Omega)$, 它是希尔伯特空间. $H^m(\Omega)$是更一般空间$W^{m,p}(\Omega)$(Sobolev空间)的特殊情况. 我们有$H^m(\Omega)=W^{m,2}(\Omega)$. 由于偏微分方程的广泛应用, $H^m(\Omega)$是最重要的一个Hilbert空间.
\end{example}
因为每一个内积空间都有范数, 所以它也有由范数定义的收敛. 这个收敛叫做强收敛
\subsubsection*{强收敛}
\begin{definition}
  内积空间$E$中的序列$(x_n)$, 如果当$n\to\infty$时有$\|x_n-x\|\to 0$, 则称序列$(x_n)$强收敛于$E$中的向量$x$.
\end{definition}
强收敛这个词是为了区别于弱收敛.
\subsubsection*{弱收敛}
\begin{definition}
  内积空间$E$中有序列$(x_n)$, 如果当$n\to \infty$时对于任意$y\in E$有$\langle x_n,y\rangle\to\langle x,y\rangle$, 则称序列$(x_n)$弱收敛于$E$中的向量$x$
\end{definition}
上述定义可以写成当$n\to\infty$时, 对于任意$y\in E$有$\langle x_n-x,y\rangle\to 0$. 我们通常用$x_n\to x$表示强收敛, $x_n\stackrel{w}{\to} x$表示弱收敛
\begin{theorem}
  强收敛必定弱收敛, 也就是说, $x_n\to x$意味着$x_n\stackrel{w}{\to}x$
\end{theorem}
\begin{proof}
  设序列$(x_n)$强收敛于$x$. 也就是说当$n\to \infty$时, 有$\|x_n-x\|\to 0$. 根据Cauchy-Schwarz不等式
  \begin{equation*}
    |\langle x_n-x,y\rangle|\leq\|x_n-x\|\|y\|\to 0\text{当$n\to\infty$}
  \end{equation*}
  这样, 当$n\to \infty$时, 对于任意$y\in E$, 有$\langle x_n-x,y\rangle\to 0$.
\end{proof}
对于内积空间$E$中任意的固定点$y$, 映射$\langle\cdot,y\rangle:E\mapsto\mathbb{C}$是$E$上的一个线性泛函. 上述定理告诉了我们对于任意$y\in E$, 这个线性泛函连续. 同样的线性泛函$\langle x,\cdot\rangle\mapsto\mathbb{C}$也连续.
\begin{theorem}
  如果$x_n\to x,y_n\to y$, 则$\langle x_n,y_n\rangle\to\langle x,y\rangle$
\end{theorem}
\begin{proof}
  若$x_n\to x,y_n\to y$, 则有
  \begin{equation*}
  \begin{split}
     |\langle x_n,y_n\rangle-\langle x,y\rangle|& \leq|\langle x_n,y_n\rangle-\langle x,y_n\rangle|+|\langle x,y_n\rangle-\langle x,y\rangle| \\
       & =|\langle x_n-x,y_n\rangle|+|\langle x,y_n-y\rangle| \\
       & \leq\|x_n-x\|\|y_n\|+\|x\|\|y_n-y\|\to 0
  \end{split}
  \end{equation*}
  其中序列$(y_n)$有界.
\end{proof}
根据这个定理, 我们立马可以获得强收敛的一个性质:
\begin{equation*}
  x_n\to x\text{意味着$\|x_n\|\to\|x\|$}
\end{equation*}
\begin{theorem}
  若$x_n\stackrel{w}{\to}x$且$\|x_n\|\to\|x\|$, 则$x_n\to x$.
\end{theorem}
\begin{proof}
  若对于所有的$y$,有$x_n\stackrel{w}{\to}x$, 我们有:
  \begin{equation*}
    \langle x_n,x\rangle\to\langle x,y\rangle,\quad n\to\infty
  \end{equation*}
  因此
  \begin{equation*}
    \langle x_n,x\rangle\to\langle x,x\rangle=\|x\|^2
  \end{equation*}
  现在, 当$n\to\infty$时
  \begin{equation*}
    \begin{split}
       \|x_n-x\|^2 & =\langle x_n-x,x_n-x\rangle \\
         & =\langle x_n,x_n\rangle-\langle x_n,x\rangle-\langle x,x_n\rangle+\langle x,x\rangle \\
         & =\|x_n\|^2-2Re\langle x_n,x\rangle+\|x\|^2\to\|x\|^2-2\|x\|^2+\|x\|^2=0
    \end{split}
  \end{equation*}
  因此序列$(x_n)$强收敛于$x$
\end{proof}
\section{正交归一系统}
向量空间中可以给定一族基$\mathcal{B}$, 使得任意向量$x\in E$能被写成$\sum_{n=1}^{m}\lambda_nx_n$, 其中$x_n\in \mathcal{B}$, $\lambda_n$是标量. 在内积空间里正交基非常重要. 无限求和可以取代有限求和$\sum_{n=1}^{m}\lambda_nx_n$, 并且线性无关的条件可以替换成正交条件.
\subsection*{正交和正交归一系统}
\begin{definition}
  内积空间$E$中的一族非零向量族$S$, 如果对于$S$中任何不同的两个元素$x,y$都有$x\perp y$, 则叫做正交系统. 除此之晚, 如果对于任意$x\in S$都有$\|x\|=1$, 则叫$S$正交归一系统.
\end{definition}
每一个正交非零向量的集合都可以归一化: 如果$S$是一个正交系统, 则$S={\frac{x}{\|x\|}:x\in S}$是一个正交归一系统.

\indent指的注意如果$x$于每一个$y_1,y_2,\dots,y_n$都正交, 则$x$与$y_1,y_2,\dots,y_n$的任意线性组合都正交. 事实上如果$y=\sum_{k=1}^{n}\lambda_ky_k$, 我们有:
\begin{equation*}
  \langle x,y\rangle=\langle x,\sum_{k=1}^{n}\lambda_ky_k\rangle=\sum_{k=1}^{n}\lambda_k\langle x,y_k\rangle=0
\end{equation*}
\begin{theorem}
  正交系统是线性无关的系统
\end{theorem}
\begin{proof}
  令$S$是一个正交系统. 假设存在$x_1,x_2\dots,x_n\in S$和$\alpha_1,\alpha_2,\dots,\alpha_n$有$\sum_{k=1}^{n}\alpha_kx_k=0$, 则
  \begin{equation*}
    0=\sum_{m=1}^{n}\langle0,\alpha_mx_m\rangle=\sum_{m=1}^{n}\langle\sum_{k=1}^{n}\alpha_kx_k,\alpha_mx_m\rangle=\sum_{m=1}^{n}|\alpha_m|^2\|x_m\|^2
  \end{equation*}
  这表明了对于任意$m\in \mathbb{N}$都有$\alpha_m=0$, 因此$x_1,\dots,x_n$是线性无关的.
\end{proof}
\subsubsection*{正交序列}
\begin{definition}
  组成正交归一系统的向量序列叫做正交归一序列.
\end{definition}
在实际中, 我们经常用全体整数$\mathbb{Z}$来表示序列指标. 正交归一序列$(x_n)$的条件可以用$Kronecker$符号去表示.
\begin{equation*}
  \langle x_m,x_n\rangle=\delta_{mn}=\begin{cases}
                                       0 & \text{如果$m\neq n$ }\\
                                       1 & \text{如果$m=n$}
                                     \end{cases}
\end{equation*}
\begin{example}
  对于向量$e_n=(0,0,\dots,1,0,\dots,\dots)$, 其中$1$在第$n$个位置.集合$S={e_1,e_2,\dots,e_n,\dots}$构成$l^2$中的正交归一系统.
\end{example}
\begin{example}
  令$\varphi_n(x)=\frac{e^{inx}}{\sqrt{2\pi}},n\in \mathbb{Z}$. 集合${\varphi_n:n\in\mathbb{Z}}$是$L^2([-\pi,\pi])$中的一个正交归一系统. 的确, 对于$m\neq n$, 我们有
  \begin{equation*}
    \langle\varphi_n,\varphi_m\rangle=\frac{1}{2\pi}\int_{-\pi}^{\pi}e^{i(m-n)x}dx=\frac{e^{i\pi(m-n)-e^{i\pi(n-m)}}}{2\pi i(m-n)}=0
  \end{equation*}
  另一方面
  \begin{equation*}
    \langle\varphi_n,\varphi_n\rangle=\frac{1}{2\pi}\int_{-\pi}^{\pi}e^{i(n-n)x}dx=1
  \end{equation*}
  因此$\langle\varphi_m,\varphi_n\rangle=\delta_{mn},m,n\in\mathbb{Z}$.
\end{example}
\begin{example}
  Legendre多项式如下定义
  \begin{equation*}
    \begin{split}
       P_0(x) & =1 \\
       P_n(x) & =\frac{1}{2^n n!}\frac{d^n}{dx^n}(x^2-1)^n,\quad n=1,2,3,\dots
    \end{split}
  \end{equation*}
  构成$L^2([-1,1])$上的正交系统. 为了方便, 我们记$(x^2-1)^n=p_n(x)$, 这样
  \begin{equation*}
    \int_{-1}^{1}P_n(x)x^mdx=\frac{1}{2^n n!}\int_{-1}^{1}p_n^{(n)}(x)x^mdx
  \end{equation*}
  对于$m<n$, 我们通过递推计算这个积分. 首先我们注意到
  \begin{equation*}
    \int_{-1}^{1}p_n^{(n)}(x)x^mdx=-m\int_{-1}^{1}p_n^{(n-1)}(x)x^{m-1}dx
  \end{equation*}
  重复这个操作直到最后推导出
  \begin{equation*}
    (-1)^m m!\int_{-1}^{1}p_n^{n-m}(x)dx=(-1)^m m![p_n^{(n-m-1)}(x)]|^1_{-1}=0
  \end{equation*}
  所以
  \begin{equation*}
    \int_{-1}^{1}P_n(x)x^mdx=0,\quad m<n
  \end{equation*}
  因为$P_m(x)$是$m$阶多项式, 这显然可以导出
  \begin{equation*}
    \langle P_n,P_m\rangle=\int_{-1}^{1}P_n(x)P_m(x)dx=0,\quad m\neq n
  \end{equation*}
  这样我们证明了Legendre多项式的正交性. 为了获得Legendre多项式的正交归一系统. 我们还需要计算$L^2([-1,1])$中$P-n$的范数:
  \begin{equation*}
    \|P_n\|=\sqrt{\int_{-1}^{1}(P_n(x))^2dx}
  \end{equation*}
  首先我们得到
  \begin{equation*}
  \begin{split}
     \int_{-1}^{1}(1-x^2)^ndx & =\int_{-1}^{1}(1-x)^n(1+x)^ndx \\
       & =\frac{n}{n+1}\int_{-1}^{1}(1-x)^{n-1}(1+x)^{n+1}dx=\cdots \\
       & =\frac{n(n-1)\cdots2\cdot1}{(n+1)(n+2)\cdots 2n}\int_{-1}^{1}(1+x)^{2n}dx \\
       & =\frac{(n!)^2 2^{2n+1}}{(2n)!(2n+1)}
  \end{split}
  \end{equation*}
  类似的过程
  \begin{equation*}
  \begin{split}
     \int_{-1}^{1}(p_n^{(n)}(x))^2 & =0-\int_{-1}^{1}p_n^{(n-1)}(x)p_n^{(n+1)}(x)dx \\
       & =\cdots \\
       & =(-1)^n\int_{-1}^{1}p_n(x)p_n^{(2n)}(x)dx \\
       & =(2n)!\int_{-1}^{1}(1-x)^n(1+x)^ndx
  \end{split}
  \end{equation*}
  其中我们有$p_n(x)$的$2n$次导数, 这样就只剩下$2n$次项. 根据以上各个式子可以得到:
  \begin{equation*}
    \int_{-1}^{1}(P_n(x))^2 dx=\frac{1}{(2^n n!)^2}(2n)!\frac{(n!)^2 2^{2n+1}}{(2n)!(2n+1)}=\frac{2}{2n+1}
  \end{equation*}
  这样多项式$\sqrt{n+\frac{1}{2}}P_n(x)$构成$L^2([-1,1])$中的正交归一系统.
\end{example}
\begin{example}
  我们用$H_n$表示$n$阶Hermite多项式.
  \begin{equation*}
    H_n(x)=(-1)^ne^{x^2}\frac{d^n}{dx^n}e^{-x^2}
  \end{equation*}
  函数$\varphi_n(x)=e^{-\frac{x^2}{2}}H_n(x)$构成$L^2(\mathbb{R})$上的正交系统. 内积为
  \begin{equation*}
    \langle\varphi_n,\varphi_m\rangle=(-1)^{n+m}\int_{-\infty}^{\infty}e^{x^2}\frac{d^n}{dx^n}e^{-x^2}\frac{d^m}{dx^m}e^{-x^2}dx
  \end{equation*}
  两边乘以$(-1)^{n+m}$
  \begin{equation}\label{Hermite polynomial}
    (-1)^{n+m}\langle\varphi_n,\varphi_m\rangle=\left[ e^{x^2}\frac{d^n}{dx^n}e^{-x^2}\frac{d^m}{dx^m}e^{-x^2}\right]^{\infty}_{-\infty}-\int_{-\infty}^{\infty}\frac{d}{dx}\left[e^{x^2}\frac{d^n}{dx^n}e^{-x^2}\right]\frac{d^{m-1}}{dx^{m-1}}e^{-x^2}dx
  \end{equation}
  在微分符号里的所有项都包含因子$e^{-x^2}$. 因此对于所有的$k\in \mathbb{N}$我们有
  \begin{equation*}
    x^ke^{-x^2}\to 0\text{当}x\to\infty
  \end{equation*}
  所以\eqref{Hermite polynomial}第一项收敛. 不断重复分部积分最后就能得到
  \begin{equation*}
    \langle\varphi_n,\varphi_m\rangle=0, n\neq m
  \end{equation*}
  为了获得正交归一系统, 我们计算范数
  \begin{equation*}
    \|\varphi_n\|^2=\int_{-\infty}^{\infty}e^{-x^2}(H_n(x))^2dx=\int_{-\infty}^{\infty}e^{-x^2}\left[e^{x^2}\frac{d^n}{dx^n}e^{-x^2}\right]^2dx
  \end{equation*}
  因为$H_n(x)$是$n$次多项式, 直接求导得到
  \begin{equation*}
    e^{x^2}\frac{d^n}{dx^n}e^{-x^2}=(-2x)^n+\cdots
  \end{equation*}
  然后在求导
  \begin{equation*}
    \frac{d^n}{dx^n}\left[e^{x^2}\frac{d^n}{dx^n}e^{-x^2}\right]=\frac{d^n}{dx^n}((-2x)^n+\cdots)=(-1)^n2^nn!
  \end{equation*}
  所以得到
  \begin{equation*}
    \|\varphi_n\|^2=2^nn!\int_{-\infty}^{\infty}e^{-x^2}dx=2^nn!\sqrt{\pi}
  \end{equation*}
  因此函数
  \begin{equation*}
    \psi_n(x)=\frac{1}{\sqrt{2^nn!\sqrt{\pi}}}e^{-\frac{x^2}{2}}H_n(x)
  \end{equation*}
  构成$L^2(\mathbb{R})$中的正交归一系统.
\end{example}
在之前的例子里, 正交函数序列是正交的但不是归一的. 尽管计算比较复杂, 但我们总是可以对函数归一化获得正交归一的函数序列. 同样的对于线性无关的函数序列, 我们也可以通过Schmidt正交化的方案获得正交归一序列. 过程如下

\indent给定内积空间中线性无关的序列$(y_n)$, 定义序列$(w_n),(x_n)$
\begin{equation*}
\begin{split}
   w_1 & =y_1\quad\quad\quad\quad\quad\quad\quad\quad x_1=\frac{w_1}{\|w_1\|}\\
    w_k & =y_k-\sum_{n=1}^{k}\langle x_n,y_k\rangle x_n,\quad x_k=\frac{w_k}{\|w_k\|}
\end{split}
\end{equation*}
序列$(w_n)$是正交的. 我们注意到
\begin{equation*}
\begin{split}
   \langle w_1,w_2\rangle & =\langle y_2-\langle x_1,y_2\rangle x_1,y_1\rangle=\langle y_2,y_1\rangle-\langle y_2,x_1\langle x_1,y_1\rangle \\
     & =\langle y_2,y_1\rangle-\frac{\langle y_2,y_1\rangle\langle y_1,y_1\rangle}{\|y_1\|^2}=0
\end{split}
\end{equation*}
假设$w_2,\dots,w_{k-1}$都是正交的, 对于任意$m<k$都有
\begin{equation*}
\begin{split}
   \langle w_k,w_m\rangle & =\langle y_k,w_m\rangle-\frac{\sum_{m-1}^{k-1}\langle y_k,w_n\rangle\langle w_n,w_m\rangle}{\|w_m\|^2} \\
     & =\langle y_k,w_m\rangle-\frac{\langle y_k,w_m\rangle\langle w_m,w_m\rangle}{\|w_m\|^2}=0
\end{split}
\end{equation*}
因此向量$w_1,w_2,\dots,w_k$是正交的.根据数学归纳法, 序列$(w_n)$是正交序列, 因此$(x_n)$是正交归一序列. 很容易检验向量$x_1,x_2,\dots,x_n$的线性组合也是$y)1,\dots,y_n$的线性组合, 反之亦然.换句话说对于任意$n\in \mathbb{N}$有$span{x_1,\dots,x_n}=span{y_1,\dots,y_n}$.

\indent在上一章节我们证明了在内积空间中任意正交向量都满足勾股定理, 显然这可以推广到任意有限个正交向量.
\begin{theorem}[勾股定理]
  如果$x_1,x_2,\dots,x_n$是内积空间中的正交向量, 则有
  \begin{equation*}
    \|\sum_{k=1}^{n}x_k\|^2=\sum_{k=1}^{n}\|x_k\|^2
  \end{equation*}
\end{theorem}
\section{张量乘积}
\chapter{基本物理图像}
\section{量子力学与经典力学}
\indent自牛顿力学以来, 经典力学不断发展,并广泛应用于动力学系统,包括与物质相互作用的电磁场. 基本思想与支配它们应用的规律形成了一个简单而优美的方案,人们不禁认为,如果对这种方案作重大修改,必然破坏其所有吸引人的特点. 尽管如此,现在发展了一种新的方案,称为量子力学,它不仅更适合描述原子尺度的现象,而且在某些方面,它比经典方案更优美,更令人满意. 这是由于新方案所包含的改变具有十分深刻的特征,而且这一改变与那些使得经典力学如此吸引人的特点并不冲突,因而新方案能够兼容经典理论学的所有这些特点.
\section{双缝干涉}
\section{波函数与状态空间}
\chapter{可观测力学量}
\section{基本物理学量}
\section{自伴算子(厄米算符)}
\section{表象变换与本征态}
\section{李乘法}
\section{基本量子化条件}
\section{不确定关系}
\chapter{量子动力学}
\section{薛定谔方程}
\subsection{能量本征方程}
\subsection{概率流守恒}
\subsection{一维well问题举例}
\section{单参数幺模群——时间演化算符}
\section{传播子与格林函数}
\section{动力学图像——薛定谔图像、海森堡图像、狄拉克图像}
\chapter{量子力学对称性}
\section{平移对称性与动量}
\section{转动对称性与角动量}
\section{李群李代数及其表示}
\section{角动量及其耦合}
\section{不可约张量算符}
\chapter{三维束缚态}
\section{渐进分析}
\section{无限深球well}
\section{有限深球well}
\section{三维谐振子}
\section{氢原子严格解}
\chapter{三维散射态}
\section{经典散射与微分截面}
\section{含时散射}
\section{$S-matrix$}
\section{Dyson公式}
\section{Born近似}
\chapter{多粒子量子系统}
\section{全同粒子与交换对称性}
\section{置换群与杨图、杨算子}
\section{二次量子化}
\section{电磁场正则量子化}
\section{Casmir效应}
\part{相对论量子力学}
\chapter{连续场基本方程}
\section{欧拉-拉格朗日方程}
\section{诺特原理与诺特流}
\section{克莱因-高登方程}
\section{克莱因-高登方程正则量子化}
\chapter{洛伦兹群}
\section{洛伦兹变换}
\section{Clifford代数的表示——$\gamma$矩阵}
\section{狄拉克方程}
\section{狄拉克方程正则量子化}
\section{洛伦兹群的分离对称性与电荷对称性——$\mathcal{CPT}$联合不变性}
\chapter{微扰理论}
\section{Gellman-Low公式与真空}
\section{Wick定理}
\section{Feynman图与费曼规则}
\subsection{$\phi^4$的费曼规则}
\subsection{QED的费曼规则}
\section{截面与$S$矩阵元}
\chapter{树图计算}
\section{$e^+e^-\rightarrow\mu^+\mu^-$}
\section{康普顿散射}
\section{库仑散射}
\chapter{圈图计算}
\section{顶角近似}
\section{辐射修正,正规化方案——PV方案}
\section{场强重整化}
\section{LSZ约化定理}
\section{光学定理}
\section{Ward-Takahashi恒等式}
\section{电荷重整化}
\part{规范理论}
\chapter{$U(1)$规范理论}
\chapter{$SU(n)$规范理论}
\section{规范理论的内禀空间与底流形}
\section{规范自由与纤维丛}
\section{规范固定与截面}
\section{Yang-Mills方程}
\section{Wilson圈}
\chapter{路径积分量子化}
\section{量子力学的路径积分}
\section{标量场的量子化}
\section{QED的量子化}
\section{旋量场的量子化}
\chapter{$Non-Abel$规范场量子化}
\section{Faddeev-Poppov路径积分量子化}
\section{鬼}
\section{BRST对称性}
\section{$Non-Abel$场的单圈修正}
\section{渐进自由}
\chapter{量子色动力学}
\section{夸克理论}
\section{胶子辐射}
\section{强子散射:轻子对}
\chapter{微扰理论}
\section{二维手征流}
\section{四维手征流}
\section{Goldstone定理}
\section{手征流反常}
\chapter{自发对称性破缺}
\section{Higgs机制}
\section{弱相互作用的Glashow-Weinberg-Salam理论}
\end{document} 